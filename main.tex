%%%%%%%%%%%%%%%%%%%%%%%%%%%%%%%%%%%%%%%%%%%%%%%%%%%%%%%%%%%%
%%% HBN-POD2 ELIFE SUBMISSION
%%%%%%%%%%%%%%%%%%%%%%%%%%%%%%%%%%%%%%%%%%%%%%%%%%%%%%%%%%%%
%%% PREAMBLE
\documentclass[9pt,lineno]{elife}
% Use the onehalfspacing option for 1.5 line spacing
% Use the doublespacing option for 2.0 line spacing
% Please note that these options may affect formatting.
% Additionally, the use of the \newcommand function should be limited.

% \usepackage[T1]{fontenc}
\usepackage{graphicx}
\usepackage{siunitx}
\usepackage[export]{adjustbox}
\usepackage{subcaption}
% \usepackage{longtable}
\usepackage{multicol}
\usepackage{xargs}  % Use more than one optional parameter in a new commands
\usepackage[colorinlistoftodos,prependcaption]{todonotes}

\graphicspath{{figures/}}
\sisetup{locale=US,group-minimum-digits=5,group-separator={,}}

\newcommandx{\alltodo}[2][1=]{\todo[author=Everyone,linecolor=red,backgroundcolor=red!25,bordercolor=red,#1]{#2}}
\newcommandx{\mctodo}[2][1=]{\todo[author=Matt,linecolor=blue,backgroundcolor=blue!25,bordercolor=blue,#1]{#2}}
\newcommandx{\sctodo}[2][1=]{\todo[author=Sydney,linecolor=red,backgroundcolor=red!25,bordercolor=red,#1]{#2}}
\newcommandx{\rokemtodo}[2][1=]{\todo[author=Ariel,linecolor=green,backgroundcolor=green!25,bordercolor=green,#1]{#2}}
\newcommandx{\arhtodo}[2][1=]{\todo[author=Adam,linecolor=purple,backgroundcolor=purple!25,bordercolor=purple,#1]{#2}}


%%%%%%%%%%%%%%%%%%%%%%%%%%%%%%%%%%%%%%%%%%%%%%%%%%%%%%%%%%%%
%%% ARTICLE SETUP
%%%%%%%%%%%%%%%%%%%%%%%%%%%%%%%%%%%%%%%%%%%%%%%%%%%%%%%%%%%%
\title{An open, analysis-ready, and quality controlled resource for pediatric brain white matter research}

\author[1,2\authfn{1}*]{Adam Richie-Halford}
\author[3,4\authfn{1}*]{Matthew Cieslak}
\author[5]{Lei Ai}
\author[6,10]{Sendy Caffarra}
\author[3,4]{Sydney Covitz}
\author[5,7]{Alexandre R. Franco}
\author[6,8,9]{Iliana I. Karipidis}
\author[2]{John Kruper}
\author[5,6]{Michael Milham}
\author[8]{B\'arbara Avelar-Pereira}
\author[7]{Ethan Roy}
\author[3,4]{Valerie J. Sydnor}
\author[7]{Jason Yeatman}
\author[11]{The Fibr Community Science Consortium}
% \author[6]{Fibr user 1}
\author[6]{Fibr user 2}
\author[6]{Fibr user 3}
\author[6]{Fibr user 4}
\author[6]{Fibr user 5}
\author[6]{Fibr user 6}
\author[6]{Fibr user 7}
\author[6]{Fibr user 8}
\author[6]{Fibr user 9}
\author[6]{Fibr user 10}
\author[6]{Fibr user 11}
\author[6]{Fibr user 12}
\author[6]{Fibr user 13}
\author[6]{Fibr user 14}
\author[6]{Fibr user 15}
\author[6]{Fibr user 16}
\author[6]{Fibr user 17}
\author[6]{Fibr user 18}
\author[6]{Fibr user 19}
\author[6]{Fibr user 20}
\author[6]{Fibr user 21}
\author[6]{Fibr user 22}
\author[6]{Fibr user 23}
\author[6]{Fibr user 24}
\author[6]{Fibr user 25}
\author[6]{Fibr user 26}
\author[6]{Fibr user 27}
\author[6]{Fibr user 28}
\author[6]{Fibr user 29}
\author[6]{Fibr user 30}
\author[6]{Fibr user 31}
\author[6]{Fibr user 32}
\author[6]{Fibr user 33}
\author[6]{Fibr user 34}
\author[6]{Fibr user 35}
\author[6]{Fibr user 36}
\author[6]{Fibr user 37}
\author[6]{Fibr user 38}
\author[6]{Fibr user 39}
\author[6]{Fibr user 40}
\author[6]{Fibr user 41}
\author[6]{Fibr user 42}
\author[6]{Fibr user 43}
\author[6]{Fibr user 44}
\author[6]{Fibr user 45}
\author[6]{Fibr user 46}
\author[6]{Fibr user 47}
\author[6]{Fibr user 48}
\author[6]{Fibr user 49}
\author[6]{Fibr user 50}
\author[6]{Fibr user 51}
\author[6]{Fibr user 52}
\author[6]{Fibr user 53}
\author[6]{Fibr user 54}
\author[6]{Fibr user 55}
\author[6]{Fibr user 56}
\author[6]{Fibr user 57}
\author[6]{Fibr user 58}
\author[6]{Fibr user 59}
\author[6]{Fibr user 60}
\author[6]{Fibr user 61}
\author[6]{Fibr user 62}
\author[6]{Fibr user 63}
\author[6]{Fibr user 64}
\author[6]{Fibr user 65}
\author[6]{Fibr user 66}
\author[6]{Fibr user 67}
\author[6]{Fibr user 68}
\author[6]{Fibr user 69}
\author[6]{Fibr user 70}
\author[6]{Fibr user 71}
\author[6]{Fibr user 72}
\author[6]{Fibr user 73}
\author[6]{Fibr user 74}
\author[6]{Fibr user 75}
\author[6]{Fibr user 76}
\author[6]{Fibr user 77}
\author[6]{Fibr user 78}
\author[6]{Fibr user 79}
\author[6]{Fibr user 80}
\author[6]{Fibr user 81}
\author[6]{Fibr user 82}
\author[6]{Fibr user 83}
\author[6]{Fibr user 84}
\author[6]{Fibr user 85}
\author[6]{Fibr user 86}
\author[6]{Fibr user 87}
\author[6]{Fibr user 88}
\author[6]{Fibr user 89}
\author[6]{Fibr user 90}
\author[6]{Fibr user 91}
\author[6]{Fibr user 92}
\author[6]{Fibr user 93}
\author[6]{Fibr user 94}
\author[6]{Fibr user 95}
\author[6]{Fibr user 96}
\author[6]{Fibr user 97}
\author[6]{Fibr user 98}
\author[6]{Fibr user 99}
\author[6]{Fibr user 100}
\author[6]{Fibr user 101}
\author[6]{Fibr user 102}
\author[6]{Fibr user 103}
\author[6]{Fibr user 104}
\author[6]{Fibr user 105}
\author[6]{Fibr user 106}
\author[6]{Fibr user 107}
\author[6]{Fibr user 108}
\author[6]{Fibr user 109}
\author[6]{Fibr user 110}
\author[6]{Fibr user 111}
\author[6]{Fibr user 112}
\author[6]{Fibr user 113}
\author[6]{Fibr user 114}
\author[6]{Fibr user 115}
\author[6]{Fibr user 116}
\author[6]{Fibr user 117}
\author[6]{Fibr user 118}
\author[6]{Fibr user 119}
\author[6]{Fibr user 120}
\author[6]{Fibr user 121}
\author[6]{Fibr user 122}
\author[6]{Fibr user 123}
\author[6]{Fibr user 124}
\author[6]{Fibr user 125}
\author[6]{Fibr user 126}
\author[6]{Fibr user 127}
\author[6]{Fibr user 128}
\author[6]{Fibr user 129}
\author[6]{Fibr user 130}
\author[6]{Fibr user 131}
\author[6]{Fibr user 132}
\author[6]{Fibr user 133}
\author[3,4\authfn{2}]{Theodore D. Satterthwaite}
\author[2,1\authfn{2}]{Ariel Rokem}

\corr{richiehalford@gmail.com}{ARH}
\corr{matthew.cieslak@pennmedicine.upenn.edu}{MC}

\affil[1]{University of Washington, eScience Institute, Seattle, Washington, 98195, USA}
\affil[2]{University of Washington, Department of Psychology, Seattle, Washington, 98195, USA}
\affil[3]{University of Pennsylvania, Department of Psychiatry, Philadelphia, Pennsylvania, 19104, USA}
\affil[4]{University of Pennsylvania, Lifespan Informatics and Neuroimaging Center, Philadelphia, Pennsylvania, 19104, USA}
\affil[5]{Child Mind Institute, Center for the Developing Brain, New York City, New York, 10022, USA}
\affil[6]{Stanford University, Graduate School of Education and Division of Developmental and Behavioral Pediatrics, Stanford, California, 94305, USA}
\affil[7]{Nathan Kline Institute for Psychiatric Research, Center for Biomedical Imaging and Neuromodulation, Orangeburg, New York, 10962, USA}
\affil[8]{Stanford University, Center for Interdisciplinary Brain Sciences Research, Department of Psychiatry and Behavioral Sciences, School of Medicine, Stanford, California, 94305, USA}
\affil[9]{University of Zurich, Department of Child and Adolescent Psychiatry and Psychotherapy, University Hospital of Psychiatry Zurich, Zurich, 8032, Switzerland}
\affil[10]{University of Modena and Reggio Emilia, Department of Biomedical, Metabolic and Neural Sciences, Via Campi 287, 41125 Modena, Italy}
\affil[11]{The Fibr Community Science Consortium}

\contrib[\authfn{1}]{These authors contributed equally to this work}
\contrib[\authfn{2}]{These authors also contributed equally to this work}

% \presentadd[\authfn{3}]{Department, Institute, Country}
% \presentadd[\authfn{4}]{Department, Institute, Country}

%%%%%%%%%%%%%%%%%%%%%%%%%%%%%%%%%%%%%%%%%%%%%%%%%%%%%%%%%%%%
%%% ARTICLE START
%%%%%%%%%%%%%%%%%%%%%%%%%%%%%%%%%%%%%%%%%%%%%%%%%%%%%%%%%%%%

\begin{document}

\maketitle

\begin{abstract}
We created resources to facilitate research on the role of brain microstructure in the
development of mental health disorders, based on openly-available diffusion MRI
(dMRI) data from the Healthy Brain Network (HBN). First, we curated the
HBN dMRI data (N=\num{2747}) into the Brain Imaging Data
Structure (BIDS) and preprocessed it according to best-practices, including denoising and correcting for motion effects, susceptibility-related
distortions, and eddy currents. 
\todo[inline]{Comment ARF: A sentence (or two) describing diffusion derivatives is missing in the abstract}
Data quality plays a key role in the analysis of dMRI. Analysis of these data requires considerations
of data quality. To scale quality control (QC) to this large dataset, we
trained a convolutional neural network through the combination of a small data
subset scored by experts and a larger set scored by web-based community
scientists. The network performs QC highly concordant with that of an expert (how much?) on a left
out set (HOW MUCH AGREEMENT?, CAN PUT SOME RESULTS HERE).
The dataset both (a) delivers a resource for transdiagnostic
research in brain connectivity and pediatric mental health and (b) serves as a paradigm
for automated QC of large datasets.
\end{abstract}

\todo[inline]{%
    Note to editing authors:

    We're targeting an eLife ``Tools and Resources'' paper for first submission.
    For eLife, the abstract should be around 150 words. Suggestions welcome.

    We use the \texttt{todonotes} package to keep track of remaining tasks and
    comments. You can add a task for Adam with the
    \texttt{\textbackslash{}arhtodo} command, a task for Matt with the
    \texttt{\textbackslash{}mctodo} command, a task for Ariel with the
    \texttt{\textbackslash{}rokemtodo} command, a task for all reviewing authors
    with the \texttt{\textbackslash{}alltodo} command, and a general comment
    with the generic \texttt{\textbackslash{}todo} command.
}
\todo[inline]{Comments look like this.}
\arhtodo[inline]{Tasks for Adam look like this.}
\rokemtodo[inline]{Tasks for Ariel look like this.}
\mctodo[inline]{Tasks for Matt look like this.}
\alltodo[inline]{Tasks for all reviewing authors look like this.}

\section{Introduction}

Childhood and adolescence are characterized by rapid dynamic change to human brain
structure and function \citep{Lebel2018-oy}. This period of development is also a time during which the
symptoms of many mental health disorders emerge \citep{Paus2008-gi}.
Understanding how individual differences in brain development relate to the onset and progression of psychopathology inevitably requires large datasets  \citep{Paus2010-qk, Fair2021-eg}.
The Healthy Brain Network (HBN) is a landmark pediatric mental health study
that is projected to include MRI images along with detailed clinical and cognitive phentoyping from \num{10000} New York
City area children and adolescents \citep{alexander2017-yc}. The HBN dataset
takes a trans-diagnostic approach and provides a broad range of phenotypic and brain
imaging data for each individual. One of the brain imaging measurements acquired 
is diffusion MRI (dMRI), which is the dominant technology for inferring the
physical properties of white matter \citep{wandell2016-qt}. The dMRI
data is openly available in its raw form through the Functional Connectomes
Project and the International Neuroimaging Data-Sharing Initiative (FCP-INDI),
spurring collaboration on open and reproducible science
\citep{Mennes2013-dl, avesani2019-ey}. However, this raw, publicly available data requires extensive processing and quality assurance before it can be fruitfully analyzed.

The analysis of a large, multi-site dMRI dataset must take into account the inevitable variability
in scanning parameters across scanning sessions.
Critical preprocessing steps, such as susceptibility 
distortion correction \citep{jones2010-ps} require additional non-DWI acquisitions and accurate metadata accompanying each
image.
A session missing an acquisition or important metadata can either be processed to the extent 
its available data allows or excluded entirely.
In addition, the quality of preprocessed data is heavily affected by 
differences in acquisition parameters \citep{yeh2019-kb} and by differences in preprocessing
steps.
Here we address these problems by meticulously curating
the HBN data according to the Brain Imaging Data Specification (BIDS)
\citep{gorgolewski2016-lh} and processing the data using the \emph{QSIPrep} \citep{cieslak2021-iq} 
BIDS App \citep{Gorgolewski2017-mb}. \emph{QSIPrep} automatically builds and executes a 
best-practices preprocessing workflow given the available BIDS data. The results include automated
QC metrics, visual reports and a description of the processing steps automatically chosen to process
each session.

In addition to requiring extensive preprocessing, dMRI data must be thoroughly checked for quality. dMRI measurements are susceptible to a variety of artifacts that affect
the quality of the signals and the ability to make accurate inferences from
them. In small studies, with few participants, it is common to thoroughly
examine the data from every participant as part of a quality control (QC)
process. However, expert examination is time consuming and is prohibitive in large
datasets such as HBN. This difficulty could be ameliorated through the
automation of QC. Given their success in other visual recognition tasks, machine learning and computer vision methods, such as convolutional deep
artificial neural networks or ``deep learning'' \citep{lecun2015deep}, are
promising avenues for automation of QC. However, one of the challenges of these new
methods is that they require a large training dataset to attain accurate
performance. In previous work, we demonstrated that deep learning can accurately
emulate expert examination of T1-weighted (T1w) brain images for QC
\citep{keshavan2019-er}. To obtain a large enough training dataset of T1w images
in our prior study, we deployed a community science application that collected quality
control scores of parts of the dataset from volunteers through a web application
\footnote{%
    While the legacy term ``citizen science'' evokes a sense of civic duty in
    scientific engagement, it can also imply a barrier for community members who
    want to contribute to science but may not be citizens of a particular
    country. In this manuscript we use the more modern term ``community
    science.''
}.
The scores were then calibrated using a gold standard expert-scored subset of
these images. A deep learning neural network was trained on the calibrated and
aggregated score, resulting in very high concordance with expert ratings on a
separate test dataset. We termed this approach ``hybrid QC'', because it combined
information from experts with information from community scientists to create a
scalable machine learning algorithm that can be applied to future data
collection.

However, the hybrid QC proof-of-concept left lingering questions about its
applicability to other datasets because it was trained on a single-site,
single-modality dataset. Here, we expand the hybrid-QC approach to a large
multi-site diffusion MRI dataset. Moreover, one of the common critiques of deep
learning is that it can learn irrelevant features of the data and does not
provide information that is transparent enough to interpret
\citep{lipton2017doctor, salahuddin2022transparency, Zech2018-ki}. To confirm
that the hybrid-QC deep learning algorithm uses meaningful features of the
diffusion-weighted images to perform accurate QC, we used machine learning
interpretation methods that pry open the ``black box'' of the neural network, thereby
highlighting the features that lead to a specific QC score
\citep{sundararajan2017axiomatic,murdoch2019definitions}.

Taken together, the combination of curated BIDS data, preprocessed images, and quality control scores generated by the deep learning algorithm provides researchers with a rich and accessible data resource. As described below, we anticipate that these HBN Preprocessed Open Diffusion Derivatives (HBN-POD2) will accelerate translational research on both normal and abnormal brain development.

\section{Results}

The aims of this study were four-fold
\begin{enumerate*}[%
    label=(\roman*),%
    before=\unskip{: },%
    itemjoin={{, }},%
    itemjoin*={{, and }}]
    \item curate the HBN MRI data into a fully BIDS-compliant MRI dataset
    \item perform state-of-the-art diffusion MRI (dMRI) preprocessing using \emph{QSIPrep}
    \item assign QC scores to each participant
    \item provide unrestricted public release of the outputs from each of these
    steps.
\end{enumerate*}
Starting with MRI data from \num{2747} HBN participants available on FCP-INDI,
we curated these data for compliance with the Brain Imaging Data Structure
(BIDS) specification \citep{gorgolewski2016-lh} and preprocessed the structural
MRI (sMRI) and diffusion MRI (dMRI) data using \emph{QSIPrep}. Participants that
could not be curated to comply with the BIDS standard or that did not have dMRI
data were excluded, resulting in \num{2136} participants with preprocessed,
BIDS-compliant dMRI data (Figure~\ref{fig:hbn-sankey}).

\begin{figure}[tbp]
    \centering
    \includegraphics[width=0.75\linewidth]{hbn-pod2-sankey.png}
    \caption{%
        {\bf HBN-POD2 data provenance}:
        Imaging data for \num{2747} participants, aged 5-21 years and collected at four
        sites in the New York City area, was made available through the
        Functional Connectomes Project and the International Neuroimaging
        Data-Sharing Initiative (FCP-INDI).
        %
        These data were curated for compliance to the BIDS specification
        \citep{gorgolewski2016-lh} and availability of imaging metadata in json
        format. \num{2615} participants met this specification.
        %
        Imaging data was preprocessed using \emph{QSIPrep} \citep{cieslak2021-iq}
        to group, distortion correct, motion correct, denoise, coregister and
        resample MRI scans. Of the BIDS curated participants, \num{2136}
        passed this step, with the majority of failures coming from participants
        with missing dMRI scans.
        %
        Expert raters assigned QC scores to \num{200} of these participants,
        creating a ``gold standard'' QC subset. Community raters then assigned
        binary QC ratings to a superset of the gold standard containing
        \num{1653} participants. An image classification algorithm was trained
        on a combination of automated QC metrics from \emph{QSIPrep} and community
        scientist reviews to ``extend'' the expert ratings to the community
        science subset. Finally, a deep learning QC model was trained on the
        community science subset to assign QC scores to the entire dataset and
        to future releases from HBN.
        %
        The HBN-POD2 dataset, including QC ratings, is openly available through
        FCP-INDI.
    }
    \label{fig:hbn-sankey}
\end{figure}

 
\subsection{Healthy Brain Network Preprocessed Open Diffusion Derivatives}

Curated BIDS data and their corresponding \emph{QSIPrep} outputs are provided in the  FCP-INDI Amazon Web Services (AWS) S3 bucket
\footnote{%
    Curated BIDS data is available at 
    \url{s3://fcp-indi/data/Projects/HBN/BIDS_curated/}
    and \emph{QSIPrep} outputs are available at
    \url{s3://fcp-indi/data/Projects/HBN/BIDS_curated/derivatives/qsiprep/}.
}. This public resource can be accessed by anyone using standard S3
access tools.

Our curation process accounts for the acquisition variability inherent in large multi-site datasets
by identifying unique \emph{variants} in the HBN 
dMRI and fieldmap acquisitions. Each session was grouped according to metadata parameters that
affect the dMRI signal (PhaseEncodingDirection, EchoTime, VoxelSize, FlipAngle, PhasePartialFourier,
NumberOfVolumes, Fieldmap availibility). We identified a total of 20 unique DWI acquisitions across
HBN-POD2, where about 5\% of acquisitions were different from the most common DWI acquisition at their
site. The specific variant of each session is provided as a column in the participant.tsv file and a summary of variants with participant counts is provided in Appendix~\ref{app:variants}.

The processed diffusion derivatives are \href{
https://qsiprep.readthedocs.io/en/latest/preprocessing.html#outputs-of-qsiprep}{
standard \emph{QSIPrep} outputs}, which contain preprocessed imaging data
along with the corresponding QC metrics.

\emph{Anatomical Data} Preprocessed images, segmentations and transforms for spatial normalization
are located in the \texttt{anat/} directory of each session. The gray matter, white matter and 
cerebrospinal fluid (\texttt{GM}, \texttt{WM}, \texttt{CSF}) probabilistic
segmentations are provided in nifti format with the \texttt{\_probtissue} suffix. The deterministic 
segmentation is in \texttt{\_dseg.nii.gz}. All images are in alignment with AC-PC-aligned 
\texttt{sub-X\_desc-preproc\_T1w.nii.gz} unless \texttt{space-MNI152NLin2009cAsym} is in their file name,
in which case they are aligned to the MNI Nonlinear T1-weighted asymmetric brain template (version 2009c). 
The spatial transform between the AC-PC T1w image and the MNI space brain is in the ITK/ANTs format 
\texttt{sub-X\_from-MNI152NLin2009cAsym\_to-T1w\_mode-image\_xfm.h5} file. The brain mask from
\texttt{ANTsBrainExtraction.sh} is included with the suffix \texttt{\_desc-brain\_mask.nii.gz}.

\emph{Diffusion Data} The preprocessed dMRI scan and accompanying metadata are located in the 
\texttt{dwi/} directory of each session. The fully-preprocessed dMRI data is named according to
the file pattern \texttt{sub-X\_space-T1w\_desc-preproc\_dwi.nii.gz}. These images all have an isotropic voxel
size of 1.7mm and are aligned in world coordinates with \texttt{anat/sub-X\_desc-preproc\_T1w.nii.gz}.
Gradient information is provided in \texttt{bval/bvec} format compatible with DIPY and DSI Studio and
the \texttt{.b} format compatible with MRtrix3. Volume-wise QC metrics including head motion parameters
are included in the \texttt{confounds.tsv} file. QC measures for the entire image series are provided
in the \texttt{ImageQC.csv} file, which includes the neighboring DWI Correlation, number of bad slices
and head motion summary statistics.
Figure~\ref{fig:metric-dist} depicts the age distribution of study participants by sex and scan site as well as pairwise distributions for the three of these automated QC metrics that were most informative in QC models described later (see Appendix~\ref{app:feature-importance} for further details).
The \texttt{desc-brain\_mask} file is a dMRI-based brain mask that
should only be used when the T1w-based brain mask is inappropriate (i.e. when no susceptibility distortion
correction has been applied).

\begin{figure}[tbp]
    {\phantomsubcaption\label{fig:metric-dist:age}}
    {\phantomsubcaption\label{fig:metric-dist:ndc-slices}}
    {\phantomsubcaption\label{fig:metric-dist:ndc-translation}}
    {\phantomsubcaption\label{fig:metric-dist:slices-translation}}
    \centering
    \includegraphics[width=\linewidth]{bundle-profiles/qsiprep-metric-distributions.pdf}
    \caption{%
        {\bf Demographic and \emph{QSIPrep} metric distributions}:
        \textbf{(a)} HBN age distributions by sex for each scanning site. Dashed lines
        indicate age quartiles.
        %
        The remaining plots show associations between \textbf{(b)} neighboring
        diffusion-weighted imaging (DWI) correlation  \citep{yeh2019-kb} and the
        number of outlier slices, \textbf{(c)} neighboring DWI correlation and maximum relative
        translation, and \textbf{(d)} the number of outlier slices and maximum
        relative translation.
        %
        The number of outlier slices is positively associated with the maximum
        relative translation, while neighboring DWI correlation is negatively associated with the other
        two metrics.
        %
        These plots are colored by age, and reveal that older participants
        generally have higher quality data.
    }
    \label{fig:metric-dist}
\end{figure}

\subsection{Quality Control}

To QC all available HBN dMRI data, we adopted a hybrid QC approach that
combines expert rating, community science, and deep learning, drawing on the
success of a previous application in assessing the quality of HBN's structural
MRI data \citep{keshavan2019-er}. This method
\begin{enumerate*}[%
    label=(\roman*),%
    before={{ }},%
    itemjoin={{; }},%
    itemjoin*={{ and }}]
    \item starts with dMRI expert raters labelling a small subset of participants,
    the ``gold standard'' dataset
    \item amplifies these labels using a community science web application to
    extend expert ratings to a much larger subset of the data, the community
    science subset
    \item trains a deep learning model on the community science subset to
    predict expert decisions on the entire dataset.
\end{enumerate*}

\subsubsection{Expert quality control}

To create a gold standard QC dataset, we first developed \emph{dmriprep-viewer},
a dMRI data viewer and QC rating web application to display \emph{QSIPrep}
outputs and collect expert ratings \citep{richie-halford2021-viewer}. Six of the
co-authors, who are all dMRI experts, rated a 200-participant subset of the
HBN-POD2 data using extensive visual examination of each participant's dMRI data,
including the diffusion weighting imaging (DWI) time series, a plot of
motion parameters throughout the DWI scan, and full 3D volumes depicting
\begin{enumerate*}[%
    label=(\roman*),%
    before={{ }},%
    itemjoin={{, }},%
    itemjoin*={{ and }}]
    \item the brain mask and $b=0$ to T1w registration
    \item a directionally encoded color fractional anisotrophy (DEC-FA) image laid over the $b=0$ volume.
\end{enumerate*}
The experts rated participants using a five-point scale with ratings of ``definitely
fail,'' ``probably fail,'' ``not sure,'' ``probably pass,'' and ``definitely
pass.'' The distribution of scores given by the experts demonstrates that the
gold standard dataset included a range of data quality (Figure~\ref{fig:expert-qc:scatter:hist}. Mean expert ratings
correlate with the three \emph{QSIPrep} automated QC metrics that were most informative for the XGB model described in the next section:
neighboring DWI correlation \citep{yeh2019-kb} (Figure~\ref{fig:expert-qc:scatter:ndc}), maximum relative
translation (Figure~\ref{fig:expert-qc:scatter:translation}), and number of outlier slices (Figure~\ref{fig:expert-qc:scatter:outliers}). The neighboring DWI correlation characterizes
the pairwise spatial correlation between pairs of DWI volumes that sample
neighboring points in $q$-space. Since lower values indicate reduced data
quality, it is reassuring that the neighboring DWI correlation correlates directly with expert rating
(Pearson CC: $0.797$). Conversely, high relative translation and a high number of
motion outlier slices reflect poor data quality and these metrics are inversely
related to mean expert rating (Pearson CC: $-0.692$ and Pearson CC: $-0.695$,
respectively).

In addition to agreeing qualitatively with \emph{QSIPrep}'s automated QC metrics
on average, the expert raters also tended to agree with each other (Figure~\ref{fig:expert-qc:irr}). We assessed
inter-rater reliability (IRR) using the pairwise Cohen's $\kappa$
\citep{di-eugenio2004-bb}, which exceeds 0.52 in all cases, and with a mean value of
0.648. In addition to the pairwise Cohen's $\kappa$, we also computed the
intra-class correlation (ICC) \citep{hallgren2012-ze} as a measure of IRR. ICC3k
is the appropriate variant of the ICC to use when a fixed set of $k$ raters each
code an identical set of participants, as is the case here. ICC3k for inter-rater
reliability among the experts was 0.930 (95\% CI: [0.91, 0.94]), which is
qualitatively considered an ``excellent'' level of IRR \citep{Cicchetti1994-fz}.
The high IRR provides confidence that the average of the expert ratings for each
image in the gold standard is an appropriate target to use for training a
machine learning model that predicts the expert scores.

\begin{figure}[tbp]
\begin{fullwidth}
    {\phantomsubcaption\label{fig:expert-qc:scatter:hist}}
    {\phantomsubcaption\label{fig:expert-qc:scatter:ndc}}
    {\phantomsubcaption\label{fig:expert-qc:scatter:translation}}
    {\phantomsubcaption\label{fig:expert-qc:scatter:outliers}}
    {\phantomsubcaption\label{fig:expert-qc:irr}}
    \begin{subfigure}{.6\linewidth}
    \centering
    \includegraphics[width=\linewidth]{community-qc/expert-qsiprep-pairplot.pdf}
    \end{subfigure}
    \begin{subfigure}{.4\linewidth}
    \centering
    \includegraphics[width=\linewidth]{community-qc/expert-raters-cohens-kappa.pdf}
    \end{subfigure}
    \caption{%
        {\bf Expert QC results}:
        Six dMRI experts rated a subset of \num{200} participants.  Experts agree
        with \emph{QSIPrep}'s automated QC metrics.  Here we show the
        distribution of mean expert QC ratings \textbf{(a)} and associations
        between the mean expert QC rating and the \emph{QSIPrep} metrics
        \textbf{(b)} neighboring diffusion-weighted imaging (DWI) correlation 
        \citep{yeh2019-kb}, \textbf{(c)} maximum relative translation, and
        \textbf{(d)} number of outlier slices. As expected, neighboring DWI correlation is directly
        correlated with expert rating while the other two metrics are inversely
        correlated with expert rating.
        %
        \textbf{(e)} Experts agree with each other. Here we show the pairwise
        Cohen's $\kappa$ measure of inter-rater reliability (see text for ICC
        calculations). The XGB model has an inter-rater reliability (quantified
        here as Cohen's $\kappa$) that is indistinguishable from the other
        raters
    }
    \label{fig:expert-qc}
\end{fullwidth}
\end{figure}

\subsubsection{Community science quality control}

Although the expert raters achieved high IRR and yielded intuitive associations
with \emph{QSIPrep}'s automated QC metrics, generating expert QC labels for the
entire HBN-POD2 dataset would be prohibitively time consuming. To assess the
image quality of the remaining participants, we deployed \emph{Fibr}
(\url{https://fibr.dev}), a community science web application in which users
assign binary pass/fail labels assessing the quality of horizontal slice DEC-FA
images overlaid on the $b=0$ image. Specifically, \emph{Fibr} users saw individual
slices or an animated sequence of ten slices taken from the entire DEC-FA volume
that the expert raters saw. The \emph{Fibr} users therefore saw only a subset of
the imaging data that the dMRI experts had access to for a given participant, but they
saw data from many more participants. In total, \num{374} community scientists provided
\num{587778} ratings for a mean of $>50$ ratings per slice (or $>200$ ratings
per participant) from \num{1653} participants. Of the community scientists, \num{145}
raters provided $>3,000$ ratings each and are included in the \emph{Fibr} Community
Science Consortium as co-authors on this paper.

There are three issues to account for when comparing \emph{Fibr} and expert QC ratings. First, the unadjusted \emph{Fibr} ratings were overly optimistic; i.e., on average,
community scientists were not as critical as the expert raters
(Figure~\ref{fig:fibr-qc:scatter:fibr}). Second, different community scientists
provide data of differing accuracy, so that some \emph{Fibr} raters are more valuable than others. Third, important information
about data quality is provided in the \emph{QSIPrep} QC metrics, which were not available to \emph{Fibr} raters. To account for rater
variability and take advantage of the information provided by \emph{QSIPrep},
we trained gradient boosted decision trees \citep{chen2016-eb} to predict expert scores, scaled to the range $[0, 1]$ and
binarized with a $0.5$ threshold,
based on a combination of community science ratings and automated \emph{QSIPrep}
QC metrics. One can think of the gradient boosting model as assigning more weight to \emph{Fibr} raters who reliably agree with the expert raters, thereby resolving the aforesaid issues with community rater accuracy. We refer to this gradient boosting model as XGB.

To clarify the contributions of the automated QC metrics and the community
science raters, we trained two additional gradient boosting models
\begin{enumerate*}[%
    label=(\roman*),%
    before=\unskip{: },%
    itemjoin={{, }},%
    itemjoin*={{ and }}]
    \item one trained only on the automated \emph{QSIPrep} QC metrics, which we
    call XGB-q
    \item one trained on only the \emph{Fibr} ratings, which we call XGB-f.
\end{enumerate*}
XGB-f may be viewed as a data-driven weighting of community scientists' ratings,
while XGB-q may be viewed as a generalization of QC metric exclusion criteria.
XGB, combining information from both \emph{Fibr} ratings and \emph{QSIPrep} QC
metric attained a cross-validated area under the receiver operating curve
(ROC-AUC) of $0.96 \pm 0.01$ on the ``gold standard,'' where the $\pm$ indicates
the standard deviation of scores from repeated $k$-fold cross-validation
(Figure~\ref{fig:fibr-qc:scatter:xgb}). In contrast, XGB-q attained an ROC-AUC of
$0.91 \pm 0.03$ and XGB-f achieved an ROC-AUC of $0.84 \pm 0.04$.
The enhanced
performance of XGB-q over XGB-f shows that community scientists alone are not as
accurate as automated QC metrics are at predicting expert ratings. And yet, the
increased performance of XGB over XGB-q demonstrates that there is additional
image quality information to be gained by incorporating community scientist input.

As a way of evaluating the quality of the XGB predictions, consider the fact
that the average Cohen's $\kappa$ between XGB and the expert raters is 0.74,
which is higher than the average Cohen's $\kappa$ between any of the other raters
and their human peers (Figure~\ref{fig:expert-qc}). This is not surprising,
given that the XGB model was fit to optimize this match, but further
demonstrates the goodness of fit of this model.

Nevertheless, this provides confidence in using the XGB scores in the next step
of analysis. When a subset of participants is coded by multiple raters and the
reliability of their ratings is meant to generalize to other participants rated by
only one coder, the single-measure ICC3 must be used. When adding XGB to the
existing expert raters as a seventh expert, $\textbf{ICC3} = 0.709 (95\% CI:
[0.66, 0.75])$. This high ICC3 value after inclusion of the XGB model justifies
using the XGB scores as the target for training an image-based deep learning
network.

\begin{figure}[tbp]
\begin{fullwidth}
    {\phantomsubcaption\label{fig:fibr-qc:scatter:fibr}}
    {\phantomsubcaption\label{fig:fibr-qc:scatter:xgb}}
    {\phantomsubcaption\label{fig:fibr-qc:roc}}
    \begin{subfigure}{.63\linewidth}
    \centering
    \includegraphics[width=\linewidth]{community-qc/fibr-rating-scatter-plot.pdf}
    \end{subfigure}
    \begin{subfigure}{.37\linewidth}
    \centering
    \includegraphics[width=\linewidth]{community-qc/xgb-roc-curve.pdf}
    \end{subfigure}
    \caption{%
        {\bf Community science predictions of the expert ratings}:
        Scatter plots showing the relationship between mean expert rating and
        both mean \emph{Fibr} rating \textbf{(a)} and XGB prediction
        \textbf{(b)}. \emph{Fibr} raters overestimate the quality of images
        compared to expert raters. But the XGB prediction compensates for this
        by incorporating automated QC metrics and weighting more valuable
        \emph{Fibr} raters.
        %
        \textbf{(c)} ROC curves for the XGB, XGB-q, and XGB-f models.
        Translucent bands represent one standard deviation from the mean
        of the cross-validation splits.
    }
    \label{fig:fibr-qc}
\end{fullwidth}
\end{figure}

\subsubsection{Automated quality control labelling through deep learning}

While the XGB ``rater'' does a good job of extending QC ratings to the entire
community science subset, this approach requires \emph{Fibr} scores; without community
science \emph{Fibr} scores, only the less accurate XGB-q
prediction can be employed. Consequently, a new, fully automated QC approach is needed that can be readily applied to new data releases from HBN.

We therefore trained a deep convolutional neural network to predict the XGB
ratings directly from \emph{QSIPrep} outputs. We modified an existing 3D
convolutional neural network (CNN) architecture \citep{zunair2020-bs}---previously
applied to the ImageCLEF Tuberculosis Severity Assessment 2019 benchmark
\citep{dicente2019clef}---to accept multichannel input generated from the
preprocessed dMRI: the $b=0$ reference diffusion image, each of the three
cardinal axis components of the DEC-FA image, and, optionally, automated QC
metrics from \emph{QSIPrep}. We trained this network on XGB scores and validated it
against the gold standard expert-scored dataset. We refer to the convolutional
neural network model trained only on imaging data as CNN-i and the model that
incorporates automated QC metrics as CNN-i+q. The two models performed nearly
identically and achieved an ROC-AUC of $0.947$ (Figure~\ref{fig:dl-qc:roc}). The
near-identical performance suggests that \emph{QSIPrep}'s automated QC metrics
provide information that is redundant with information available in the imaging
data. Both CNN-i and CNN-i+q outperformed XGB-q, which was trained only on
automated QC metrics, but both modestly underperformed relative to the full XGB model,
that uses \emph{Fibr} scores in addition to the \emph{QSIPrep} QC scores.

The openly available HBN-POD2 data released with this paper provides four QC ratings: the mean expert QC ratings, XGB-q and
XGB predicted scores, as well as the CNN-i predicted score. However, we treat the CNN-i
score as the definitive QC score because it is available for all participants,
can be easily calculated for new participants in future HBN releases, and is more
accurate than XGB-q in predicting expert ratings in the ``gold standard'' report set. When we refer to a participant's QC score without
specifying a generating model, the CNN-i score is assumed.
Figure~\ref{fig:dl-qc} depicts the distribution of these QC scores by age
(Figure~\ref{fig:dl-qc:joint}), sex (Figure~\ref{fig:dl-qc:hist:sex}), and scanning site
(Figure~\ref{fig:dl-qc:hist:site}). QC distributions are similar for each scan site
and for male and female participants \footnote{%
    Responses for the sex variable in HBN phenotypic data are limited to
    ``male'' and ``female.''
}.

\begin{figure}[tbp]
    {\phantomsubcaption\label{fig:dl-qc:roc}}
    {\phantomsubcaption\label{fig:dl-qc:joint}}
    {\phantomsubcaption\label{fig:dl-qc:hist:sex}}
    {\phantomsubcaption\label{fig:dl-qc:hist:site}}
    \begin{subfigure}{.5\linewidth}
    \centering
    \includegraphics[width=\linewidth]{deep-learning-qc/dl_roc_auc_curve.pdf}
    \end{subfigure}
    \begin{subfigure}{.5\linewidth}
    \centering
    \includegraphics[width=\linewidth]{bundle-profiles/qc-age-jointplot.pdf}
    \end{subfigure}
    \begin{subfigure}{\linewidth}
    \centering
    \includegraphics[width=\linewidth]{bundle-profiles/qc-hist.pdf}
    \end{subfigure}
    \caption{%
        {\bf Deep learning QC scores}:
        \textbf{(a)} ROC curves for two deep learning models: one trained with
        additional automated QC metrics from \emph{QSIPrep} (blue) and one
        trained without (orange). The models performed roughly identically,
        indicating the QC metrics provide information that is redundant to
        the imaging. Both outperformed the XGB-q predictions, indicating the
        added value of the diffusion weighted images. However, both models
        underperformed the XGB predictions, which also incorporate
        information from \emph{Fibr} ratings for each scan. The error bands
        represent one standard deviation from the mean of the cross-validation
        splits.
        %
        \textbf{(b)} Joint distributions showing a strong direct association
        between age and QC score (Pearson CC: $0.31$). This likely reflects the well-known negative
        association between age and head motion in pediatric neuroimaging.
        The dots encode the mean QC score for each year of age with error bands
        representing a bootstrapped 95\% confidence interval. The line depicts
        a linear regression relating age and QC score with translucent bands
        encoding a bootstrapped 95\% confidence interval.
        %
        Histograms showing the relationship between participants QC scores and
        their sex \textbf{(c)} and scan site \textbf{(d)}. QC distributions are independent
        of sex and scanning site.
    }
    \label{fig:dl-qc}
\end{figure}

\subsubsection{Attribution masks for the deep learning classifier}

We generated post-hoc attribution maps that highlight regions of the input
volume that are relevant for the QC score. The integrated gradient method
\citep{sundararajan2017axiomatic} is a gradient-based attribution method
\citep{ancona2019gradient} that aggregates gradients for synthetic images
interpolating between a baseline image and the input image. It has been used to
interpret deep learning models applied to retinal imaging in diabetic
retinopathy \citep{sayres2019using} and glaucoma \citep{Mehta2021-zp} prediction,
as well as in multiple sclerosis prediction from brain MRI
\citep{wargnier-dauchelle2021interpretable}. Our goal is to confirm that the
CNN-i model is looking at the same features that an expert rater would look at,
thereby bolstering the decision to apply it to new data.

Figure~\ref{fig:ig} shows attribution maps for example participants from each
confusion class: true positive, true negative, false positive, and false
negative. The columns correspond to the different channels of the deep learning
input volume: the $b=0$ reference image and the DEC-FA in the $x$, $y$, and $z$
directions. The blue voxels indicate positive attribution, that is, data that supports a passing QC classification. Conversely, the red voxels indicate negative attribution, data that supports a failing QC classification.
The true positive map indicates that the network is looking at the
entire brain rather than focusing on any one anatomical region
(Figure~\ref{fig:ig:true-pos}). Moreover, the model identifies white matter
fascicles that travel along the direction of the input channel: lateral for $x$,
anterior-posterior for $y$, and superior-inferior for $z$. The true negative
attribution map (Figure~\ref{fig:ig:true-neg}) reveals that when the reference
$b=0$ volume contains motion artifacts, such as banding, the network ignores the
otherwise positive attributions for the clearly identifiable white matter tracts
in the DEC-FA channels. The false positive map (Figure~\ref{fig:ig:false-pos})
and the false negative map (Figure~\ref{fig:ig:false-neg}) should be interpreted
differently since they come from low confidence predictions; the probability of
passing hovered on either side of the pass/fail threshold. For example, in the
false positive case, the network is confused enough that it treats voxels that
are outside of the brain as informative as voxels in the major white matter
bundles.

\begin{figure}[tbp]
\begin{fullwidth}
    \centering
    \adjustbox{minipage=3.5em}{\subcaption{true positive}\label{fig:ig:true-pos}}%
    \begin{subfigure}{\dimexpr\linewidth-3.5em\relax}
    \centering
    \includegraphics[width=\linewidth]{deep-learning-qc/attribution-maps-true-pos.pdf}
    \end{subfigure}
    \adjustbox{minipage=3.5em}{\subcaption{true negative}\label{fig:ig:true-neg}}%
    \begin{subfigure}{\dimexpr\linewidth-3.5em\relax}
    \centering
    \includegraphics[width=\linewidth]{deep-learning-qc/attribution-maps-true-neg.pdf}
    \end{subfigure}
    \adjustbox{minipage=3.5em}{\subcaption{false positive}\label{fig:ig:false-pos}}%
    \begin{subfigure}{\dimexpr\linewidth-3.5em\relax}
    \centering
    \includegraphics[width=\linewidth]{deep-learning-qc/attribution-maps-false-pos.pdf}
    \end{subfigure}
    \adjustbox{minipage=3.5em}{\subcaption{false negative}\label{fig:ig:false-neg}}%
    \begin{subfigure}{\dimexpr\linewidth-3.5em\relax}
    \centering
    \includegraphics[width=\linewidth]{deep-learning-qc/attribution-maps-false-neg.pdf}
    \end{subfigure}
    \caption{%
        {\bf Integrated gradients attribution maps for the deep learning classifier}:
        Each column depicts a different channel of the input tensor: the $b=0$
        DWI volume and the DEC-FA images in the $x$, $y$, and $z$ directions.
        The first three columns show an axial slice while the last column
        shows a coronal slice. Blue voxels indicate positive attribution (i.e.
        evidence for passing the participant), while red voxels indicate negative
        attribution (i.e. evidence for QC failure). The underlying grayscale
        depicts the input channel. Each row depicts a representative participant
        from each confusion class:
        %
        \textbf{(a)} Attribution maps for a true positive prediction. The model
        looks at the entire brain and focuses on known white matter bundles in
        the DEC-FA channels. In particular, it focuses on lateral bundles in the
        $x$ direction, anterior-posterior bundles in the $y$ direction, and
        superior-inferior bundles in the $z$ direction.
        %
        \textbf{(b)} Attribution maps for a true negative prediction. The model
        focuses primarily on the $b=0$ channel, suggesting that it ignores
        DEC-FA when motion artifacts like banding are present.
        %
        \textbf{(c)} Attribution maps for a false positive prediction. Both the
        false positive and negative predictions were low confidence predictions.
        This is reinforced by the fact that the model views some voxels that are
        outside of the brain as just as informative as those in major white
        matter tracts.
        %
        \textbf{(d)} Attribution maps for a false negative prediction. The model
        fails to find long-range white matter tracts in the anterior-posterior
        and lateral directions. We also speculate that the model expects
        left-right symmetry in the DEC-FA channels and assigns negative
        attribution to asymmetrical features. }
    \label{fig:ig}
\end{fullwidth}
\end{figure}

\subsection{QC prediction models generalize to new sites \ldots sometimes}

\arhtodo[inline]{Add site generalization results}

\subsection{Quality control improves inference}

To demonstrate the effect that quality control has on inference, we analyzed
tract profile data derived from HBN-POD2 data. Tract profiling
\citep{yeatman2012-rc,jones2005pasta,colby2012along,odonnell2009tract,
kruper2021evaluating} is a subset of tractometry
\citep{jones2005pasta,bells2011tractometry}, which uses the results of dMRI
tractography to quantify properties of the white matter along major pathways.
Tract-profiling retains the values of diffusion metrics along the trajectory of
each bundle of tractography streamlines, rather than computing summary
statistics summarized at the level of each bundle. In
Figure~\ref{fig:qc-profiles:md}, we plot mean diffusivity tract profiles grouped
into four QC bins along the length of twenty-four bundles: While some bundles,
such as the cingulum cingulate (CGC) and the inferior longitudinal fasciculus
(ILF), appear insensitive to QC score, others, such as the uncinate (UNC) and
the orbital portion of the corpus callosum, exhibit strong differences between
QC bins. In most bundles, low QC scores tend to flatten the profile, indicating that mean diffusivity appears artifactually homogeneous across the bundle.

\begin{figure}[tbp]
\begin{fullwidth}
    \centering
    \includegraphics[width=\linewidth]{bundle-profiles/qc-bins-dki-md.pdf}
    \caption{%
        {\bf MD bundle profiles show large QC group differences}:
        MD profiles binned by QC score in twenty-four major while matter
        bundles.  The $x$-axis represents distance along the length of the fiber
        bundle.
        %
        The left and right uncinate bundles are the most sensitive
        to QC score. Generally, QC score tends to flatten bundle profiles.
        %
        Error bands represent bootstrapped 95\% confidence intervals. Bundle
        abbreviations for lateralized bundles contain a trailing ``L'' or ``R''
        indicating the hemisphere. Bundle abbreviations:
        inferior fronto-occipital fasciculus (IFO),
        uncinate (UNC),
        thalamic radiation (ATR),
        corticospinal (CST),
        arcuate (ARC),
        superior longitudinal fasciculus (SLF).
        inferior longitudinal fasciculus (ILF),
        cingulum cingulate (CGC),
        orbital corpus callosum (Orbital),
        anterior frontal corpus callosum (AntFrontal),
        superior frontal corpus callosum (SupFrontal),
        motor corpus callosum (Motor),
        superior parietal corpus callosum (SupParietal),
        temporal corpus callosum (Temporal),
        post-parietal corpus callosum (PostParietal), and
        occipital corpus callosum (Occipital).
    }
    \label{fig:qc-profiles:md}
    \figsupp[FA bundle profiles]{%
        {\bf FA bundle profiles binned by QC score}:
        FA profiles binned by QC score in twenty-four major while matter
        bundles. The $x$-axis represents distance along the length of the fiber
        bundle. Error bands represent bootstrapped 95\% confidence intervals.
        Bundle abbreviations are as in Figure~\ref{fig:qc-profiles:md}
    }{%
        \includegraphics[width=\linewidth]{bundle-profiles/qc-bins-dki-fa.pdf}
    }\label{figsupp:qc-profiles:fa}
\end{fullwidth}
\end{figure}

\begin{wrapfigure}{r}{.5\textwidth}
    \includegraphics[width=\hsize]{age-prediction/qc_sweep.pdf}
    \caption{%
        {\bf Imposing a QC cutoff improves age prediction}:
        Cross validated $R^2$ scores (left axis, blue dots) from an age
        prediction model increase after screening participants by QC score. We see
        the most dramatic increase in $R^2$ after imposing even the lowest
        cutoff of $0.05$. Thereafter, the $R^2$ scores trend upward until a
        cutoff of $\sim 0.95$, where the training set size (right axis, orange
        line) becomes too small to sustain model performance. The error bands
        represent a bootstrapped 95\% confidence interval.
    }
    \label{fig:age-prediction}
\end{wrapfigure}

The effect of QC score on white matter bundle profiles indicates that researchers
using HBN-POD2 should incorporate QC in their analyses, either by applying a QC cutoff
when selecting participants or by explicitly adding QC score to their inferential
models. Failure to do so may cause spurious associations or degrade predictive
performance. To demonstrate this, we select participant age as a representative phenotypic benchmark because
\begin{enumerate*}[%
    label=(\roman*),%
    before=\unskip{ },%
    itemjoin={{, }},%
    itemjoin*={{ and }}]
    \item it operates on a natural scale with meaninsful units
    \item despite the unique methodological challenges it presents for biomarker identification \cite{nelson2020biomarkers}, brain age prediction may be diagnostic of overall brain health \citep{franke2010estimating, cole2019brain, richie-halford2021multidimensional}.
\end{enumerate*}
We observed the effect of varying QC cutoff on the predictive performance of an age
prediction model. Cross-validated $R^2$ scores for an age prediction model
varies depending on the QC cutoff (Figure~\ref{fig:age-prediction}). An initial large improvement was achieved by excluding the 200 participants with the lowest QC scores, followed by a gradual increase in performance. Finally, when a large number of participants is excluded, performance deteriorated again.

\section{Discussion}

We present HBN-POD2, one of the largest child and adolescent diffusion imaging
datasets with preprocessed derivatives that is currently openly available. The dataset
was designed to comply with the best practices of the field. For example, it
complies with the current draft of the BIDS diffusion derivative specification
\citep{Pestilli2021}. It will grow continuously as the HBN study acquires more
data, eventually reaching its \num{10000} participant goal. Here we summarize the
contributions of this work, starting from the potential impact of the range of
resources we have created and ending with implications for processing and
quality control of other large-scale datasets.

\subsection{Preprocessing and quality control increase the impact of openly-available data}

The most immediate contribution of this work is a large analysis-ready dMRI
data resource, openly accessible to the public. In the past decade, projects such as
the Human Connectome Project (HCP) \citep{van-essen2013-oi}, UK Biobank
\citep{miller2016-mq}, ABCD \citep{jernigan2018-my}, and CamCAN
\citep{taylor2017-or,shafto2014-ld} and of course FCP-INDI (which includes HBN)
\citep{Mennes2013-eu} have ushered a culture of data sharing in open big-data
neuroscience. The adoption and reuse of these datasets reduces or eliminates the
data collection burden on downstream researchers. Some projects, such as the HCP
\citep{glasser2013-lo}, also provide preprocessed derivatives, further reducing
researchers' burden and extending the benefits of data-sharing from data
collection to preprocessing and secondary analysis. Following the example of the
HCP, HBN-POD2 provides analysis-ready dMRI derivatives. This avoids duplication
of and heterogeneity across the preprocessing effort while also ensuring a minimum standard of data quality
for HBN researchers.
We also provide the CuBIDS variant annotation in the participants.tsv file, allowing researchers to account for the imaging heterogeneity inherent in a dataset of this size.

The data is amenable to many different analyses, including
tractometry \citep{yeatman2012-rc, yeatman2018browser, kruper2021evaluating}, graph theoretical analysis \citep{yeh2020-nu},
and combinations with functional MRI data and other data types for the same
participants. The availability of standardized preprocessed diffusion data will
allow researchers to create and test hypotheses on the white matter properties
underlying behavior and disease, from reading and math acquisition to childhood
adversity and mental health. As such, this dataset will accelerate discovery at
the nexus of white matter microstructure and neurodevelopmental and learning
disorders.

In large developmental datasets, it is critically important to perform accurate and reliable QC of
the data.
QC is associated not just with age, but with many
phenotypic variables of interest in cognition and psychopathology \cite{siegel2017quality}.
HBN-POD2 provides four separate QC scores alongside its large dataset
of pediatric neuroimaging diffusion derivatives, paving the way for users of the
data to incorporate considerations of data quality into their analysis of the
processed data. Unsurprisingly, QC scores are strongly correlated with
age (Figure~\ref{fig:dl-qc}). This accords with the negative association
between head motion and age in developmental studies, which is well established both in general
\citep{power2012spurious,satterthwaite2012impact,fair2012distinct,yendiki2014spurious}
and specifically for resting-state fMRI in the HBN dataset \citep{alexander2017-yc}.

We demonstrate the impact of QC in a benchmark age prediction task
(Figure~\ref{fig:age-prediction}). In this case, the increase in model performance
from imposing a QC cutoff is intuitive: we know from
Figure~\ref{fig:qc-profiles:md} that participants with low QC scores have
reduced MD, but MD also decreases as participants mature
\citep{yeatman2014lifespan,richie-halford2021multidimensional}. Eliminating
participants with low QC therefore removes the ones who may look artificially older
from the analysis, improving overall performance. The most noticeable
improvement in performance comes after imposing the most modest cutoff of
$0.05$, suggesting that inferences may benefit from \emph{any} QC screening. On
the other hand, QC screening inherently introduces a tradeoff between the desire
for high quality data and the desire for a large sample size. In this case,
after a QC cutoff of around $0.9$, the training set size is reduced such that it
degrades predictive performance. Importantly, we do not expect the sensitivity
analysis of an age prediction model to generalize to other analyses and therefore
recommend that researchers using HBN-POD2 choose the most appropriate QC cutoff for their research question and consider including QC score as a model covariate in their analyses.
This recommendation also justifies the use of ROC-AUC to compare the XGB and CNN classifier families because ROC-AUC aggregates classifier performance across all possible thresholds.

\subsection{Site generalization}

\arhtodo[inline]{Add site generalization experiments.}

\subsection{Automated quality control: a tension between scalability and interpretability}

The predictive
performance of the CNN-i model (Figure~\ref{fig:dl-qc:roc}) gives
us confidence that it could accurately classify unseen data, justifying its
extension to the entire HBN-POD2 dataset and to future releases. However, it
does not explain its decisions. As deep learning models have been increasingly
applied to medical image analysis, there is an evolving interest in the
interpretability of these models \citep{salahuddin2022transparency, lipton2017doctor,
Zech2018-ki, Ghassemi2021-zg}. While an
exhaustive interpretation of deep learning QC models is beyond the scope of this
work, we provided a preliminary qualitative interpretation of the CNN-i model
(Figure~\ref{fig:ig}) that demonstrates the intuitive nature of its decisions.

Nevertheless, we recognize that decisions about QC inclusion must balance accuracy, interpretability, generalization to new data, and scalability to ever larger datasets. We therefore provide three additional scores
\begin{enumerate*}[%
    label=(\roman*),%
    before=\unskip{: },%
    itemjoin={{, }},%
    itemjoin*={{, and }}]
    \item the mean expert QC score for the 200 participants in the gold standard
    dataset
    \item the scores predicted by the XGB model, which outperformed all other models
    when evaluated against the gold standard ratings, but which are only available
    for participants that have community science scores
    \item the scores predicted by the XGB-q model, which underperformed the deep
    learning generated scores, but which rely only on the automated QC metrics
    output by \emph{QSIPrep}.
\end{enumerate*}
We view the XGB-q scores, which are available for all participants, as a more interpretable and scalable fallback because the XGB-q model ingests \emph{QSIPrep} output without any further postprocessing.
Because the XGB-q model most readily generalizes to other \emph{QSIPrep} outputs, we package it as an independent QC service in the QSIQC software package \cite{richiehalford2022qsiqc}, available both as a docker image at \texttt{ghcr.io/richford/qsiqc} and as a Streamlit app at \url{https://share.streamlit.io/richford/qsiqc/main/app.py}.
The decision to use a more interpretable but slightly less
performant method of generating QC scores was also advocated by
\cite{tobe2021longitudinal}, who noted that the Euler number of T1-weighed
images \cite{rosen2018quantitative} in the NKI-Rockland dataset can reliably predict scores generated with
\emph{Braindr}, the community science application developed in our previous work
\citep{keshavan2019-er}.

We also note that the issue of algorithmic impact in choosing a QC method is not
exclusive to the deep learning model. We have chosen models that most reliably
reproduce the gold standard ratings, but a reliable algorithm might still
negatively influence researcher's decisions. For example, excluding participants
by QC score could spur them to exclude populations deserving of study, as when
QC score is highly correlated with age or socio-economic status. We therefore
caution researchers to examine interactions between the QC scores we provide and
their phenotype of interest.

More generally, QC in the dataset that we have produced is fundamentally
anchored to the decisions made by the expert observers. While Cohen's $\kappa$
between some pairs of experts can be as low as 0.52, IRR quantified across all
of the experts with ICC3k is excellent. Nevertheless, it is possible that
improvements to the final QC scores could be obtained through improvements to
IRR, or by designing a more extensive expert QC protocol. The tradeoff between
more extensive QC for each participant and more superficial QC on more participants was
not explored in this study, but could also be the target for future research.

\subsection{Transparent pipelines provide an extensible baseline for future methods development}

While the primary audience of HBN-POD2 is researchers in neurodevelopment who
will use the dMRI derivatives in their studies, other researchers may use
HBN-POD2 to develop new preprocessing algorithms or quality control methods. In
this respect, HBN-POD2 follows \cite{avesani2019-ey}, who
recognized the diverse interests that different scientific communities have in
reusing neuroimaging data and coined the term \emph{data upcycling} to promote
multiple-use data sharing for purposes secondary to those of the original
project. Complementing the approach taken in Avesani et al.'s work, which
provided dMRI from a small number of participants preprocessed with many pipelines,
HBN-POD2 contains many participants, all processed with a single state of the art
pipeline, \emph{QSIPrep}. For researchers developing new preprocessing
algorithms, HBN-POD2 provides a large, openly available baseline to which they
can compare their results.

Similarly, neuroimaging QC methods developers will benefit from a large
benchmark dataset of expert, community science, and automated QC ratings, with
which to pilot new methods. Importantly, the architecture and parameters of the
deep learning network used for QC are also provided as part of this work,
allowing application of this network to future releases of HBN data, and
allowing other researchers to build upon our efforts. Indeed, in this work, we
have extended our previous work on what we now call ``hybrid QC''. This
approach, which we originally applied to the first two releases of the HBN
T1-weighted data \citep{keshavan2019-er} (using the \emph{Braindr} web app:
\url{https://braindr.us}) was extended here in several respects.
First, the \emph{Braindr} study used a smaller dataset of approximately 700
participants, while we extended this approach to more than \num{2000} participants.
Second, \emph{Braindr} relied on approximately \num{80000} ratings from
\num{261} users. Here, we received more than \num{500000} ratings from
\num{374} community scientists. As our understanding of the role of
community scientist contributions has evolved, we decided that we would
include as collective co-authors community scientists who contributed more
than \num{3000} ratings \citep{Ward-Fear2020-zq}.
Third, \emph{Braindr} used data from only a single site. Here, multi-site
data was used. This opens up multiple possibilities for deeper exploration of
between-site quality differences, and also for harmonization of QC across
sites.
Last, the most challenging extension of hybrid QC from \emph{Braindr} to
this study entailed developing an approach that would encompass multi-volume
dMRI data. On the one hand, this meant that the task performed by the expert
observers was more challenging, because it required examination of the full
dMRI time-series for every scan. To wit, expert inter-rater reliability was
considerably higher for the T1-weighted only data in \cite{keshavan2019-er}
than for the dMRI data used (Figure~\ref{fig:expert-qc:irr}).
On the other hand, it also meant that the 4D data had to be summarized into
2D data to be displayed in the \emph{Fibr} web application. This was
achieved by summarizing the entire time-series as a DEC-FA + $b=0$ image and
presenting community scientists with animated sections of these images that
showed how the data extended over several horizontal slices.
In addition, the extension to 4D data required developing new deep learning
architectures for analysis of 4D images, including upstream contributions to
\emph{Nobrainer}, a community-developed software library for deep learning
in neuroimaging data \citep{nobrainer}.
These extensions demonstrate that the hybrid QC approach generalizes very well
to a variety of different circumstances. Future applications of this approach
could generalize to functional MRI data, as well as other large datasets from
other kinds of measurements and other research domains.

\subsection{Future work and open problems}

The HBN study plans to acquire imaging data for \num{10000} participants,
necessitating future data releases. Since future releases of HBN will also require future releases of HBN-POD2, a plan for these is essential. This is
a general issue affecting multi-year neuroimaging projects for which derivative
data is being released before study completion. The use of \emph{QSIPrep},
\emph{cloudknot} and the containerization of the QC score assignment process
facilitate running the exact pipeline described in this paper on newly released
participants. However, this approach is somewhat unsatisfactory because it fails to
anticipate improvements in preprocessing methodology. That is, what should we do
when \emph{QSIPrep} is inevitably updated between HBN releases? Enforce
standardization by using an outdated pipeline or use state-of-the-art
preprocessing at the expense of standardized processing between releases?
Because the use of \emph{cloudknot} and AWS Spot Instances renders preprocessing
fast and relatively inexpensive, we propose a third way: if improvements to the
preprocessing pipeline are available with a new HBN release, we plan to execute
the improved pipeline on the entire HBN dataset, while preserving the previous
baseline release in an archived BIDS derivative dataset.

Undertaking the processing and QC effort to generate HBN-POD2 required
construction and deployment of substantial informatics infrastructure, including
tools for cloud computing, web applications for expert annotation and for
community science rating and analysis software. All of these tools are provided
openly, so that this approach can be generalized even more widely in other
projects and in other scientific fields.

\rokemtodo[inline]{%
    After we run the site generalization experiments, speculate about
    generalization to other datasets.
}

\subsection{Ideas and Speculation}

\alltodo[inline]{%
    We are targeting eLife for first submission and eLife now allows an
    \href{https://elifesciences.org/inside-elife/e3e52a93/elife-latest-including-ideas-and-speculation-in-elife-papers}{Ideas
    and Speculation} subsection within the discussion section. This is a spot
    for authors to ``freely speculate about the meaning and implications of
    their data.'' It is reviewed but only for ``factual inaccuracies, clarity
    and speculation not germane to the paper.'' We should take advantage of this
    opportunity. Insert your speculations here.
}
\todo[inline]{Comments from Sendy:
Can we get these QC scores for other HBN datasets (e.g., fMRI, EEG, behavioral)?
Would it make sense to have a global QC for each HBN participant that takes into account different techniques? How well can the QCs of one modality predict the QCs of a different modality on the same participants?
Would it be possible to propose a new way to evaluate the reliability of a scientific finding based on its dependence on QC scores?}

\rokemtodo[inline]{Speculate about multiverse analysis, garden of forking paths, etc.}

\section{Methods and Materials}

To facilitate replicability, Jupyter notebooks \citep{kluyver2016jupyter} and
Dockerfiles \citep{merkel2014docker} necessary to reproduce the methods described
herein are provided in the HBN-POD2 GitHub repository at
\url{https://github.com/richford/hbn-pod2-qc}. The specific version of the
repository used in this study is documented in \cite{richiehalford2022hbnpod2qc}.
The \texttt{make} or \texttt{make help}
commands will list the available commands and \texttt{make build} will build the
requisite Docker images to analyze HBN-POD2 QC data. In order to separate data
from analysis code \citep{Wilson2017-rj}, we provide intermediate data necessary
to analyze the QC results in an OSF \citep{Foster-MSLS2017-rl} project
\citep{hbn-pod2-osf}, which can be downloaded using the \texttt{make data}
command in the root of the HBN-POD2 GitHub repository. Most of the code in this
repository uses Pandas \citep{mckinney-proc-scipy-2010,reback2020pandas}, Numpy
\citep{harris2020array}, Matplotlib \citep{hunter2007matplotlib}, and Seaborn
\citep{waskom2021seaborn}.

\subsection{Inputs}

Inputs for this study consisted of MRI data from the Healthy Brain Network
pediatric mental health study \citep{alexander2017-yc}, containing dMRI data from
\num{2747} participants aged 5-21 years. These data were measured using a
\qty{1.5}{\tesla} Siemens mobile scanner on Staten Island (SI) and three fixed
\qty{3}{\tesla} Siemens MRI scanners at sites in the New York area: Rutgers
University Brain Imaging Center (RUBIC), the CitiGroup Cornell Brain Imaging
Center (CBIC), and the City University of New York Advanced Science Research
Center (CUNY). Informed consent was obtained from each participant aged 18 or
older. For participants younger than 18, written consent was obtained from their
legal guardians and written assent was obtained from the participant. Voxel
resolution was \qty{1.8}{\mm} $\times$ \qty{1.8}{\mm} $\times$ \qty{1.8}{\mm} with \num{64} non-colinear
directions measured for each of $b=1000$ \unit{\second \per \mm^{2}} and
$b=2000$ \unit{\second \per \mm^{2}}.
\todo[inline]{Comment VJS: We should mention that it is multi-shell data in the Intro/Results. Also could provide the b-values and number of gradients per b-value).}

\subsection{BIDS curation}

We curated the imaging metadata for \num{2615} of the \num{2747} currently
available HBN participants. Using dcm2bids and custom scripts, we conformed the data
to the Brain Imaging Data Structure (BIDS; \citep{gorgolewski2016-lh})
specification. The BIDS-curated dataset is available on FCP-INDI and can be
accessed via AWS S3 at \url{s3://fcp-indi/data/Projects/HBN/BIDS_curated/}.

\todo[inline]{We mention dcm2bids and ``custom scripts.'' Ask Alex which version of dcm2bids and which custom scripts. Can we put them on GitHub?}

After conforming the data to BIDS, we used the ``Curation of BIDS'' (CuBIDS) \citep{sydney-covitz2022-cubids} package to identify unique combinations, or ``variants'' of imaging parameters in the curated dataset.
CuBIDS is a Python-based software package that provides a sanity-preserving workflow to help users reproducibly parse, validate, curate, and understand heterogeneous BIDS imaging datasets. CuBIDS includes a robust implementation of the BIDS Validator that scales to large samples and incorporates DataLad \cite{halchenko2021datalad}, a distributed data management system, to ensure reproducibility and provenance tracking throughout the curation process. CuBIDS tools also employ agglomerative clustering to identify the aforementioned variants of imaging parameters. Users may then test BIDS-Apps on a subset of participants that represent the full range of acquisition parameters that are present. These variants are listed in the participants.tsv file in the BIDS-curated dataset.

\subsection{Preprocessing}

We performed dMRI preprocessing on \num{2136} participants, using \emph{QSIPrep}
\citep{cieslak2021-iq} 0.12.1, which is based on \emph{Nipype} 1.5.1
\citep{nipype1,nipype2}, RRID:SCR\_002502. \emph{QSIPrep} a robust and scalable
pipeline to group, distortion correct, motion correct, denoise, coregister and
resample MRI scans. In total, \num{417} participants failed this preprocessing
step, largely due to missing dMRI files. 
In keeping with the BIDS specification,
the preprocessed dataset is available as a derivative dataset within the
BIDS-curated dataset and can be access on AWS S3 at
\url{s3://fcp-indi/data/Projects/HBN/BIDS_curated/derivatives/qsiprep/}.
\emph{QSIPrep} fosters reproducibility by automatically generating thorough
methods boilerplate for later use in scientific publications, which we use for
the remainder of this subsection to document each preprocessing step.

\begin{itemize}

\item {\it Anatomical data preprocessing}
The T1-weighted (T1w) image was corrected for intensity non-uniformity (INU)
using \texttt{N4BiasFieldCorrection} \citep{n4} (ANTs 2.3.1), and used as
T1w-reference throughout the workflow. The T1w-reference was then skull-stripped
using \texttt{antsBrainExtraction.sh} (ANTs 2.3.1), using OASIS as target
template. Spatial normalization to the ICBM 152 Nonlinear Asymmetrical template
version 2009c \citep{mni}, RRID:SCR\_008796 was performed through nonlinear
registration with \texttt{antsRegistration} \citep{ants}, ANTs 2.3.1,
RRID:SCR\_004757, using brain-extracted versions of both T1w volume and
template. Brain tissue segmentation of cerebrospinal fluid (CSF), white-matter
(WM) and gray-matter (GM) was performed on the brain-extracted T1w using
\texttt{FAST} \citep{fsl-fast}, FSL 6.0.3:b862cdd5, RRID:SCR\_002823.

\item {\it Diffusion data preprocessing}

Any images with a $b$-value less than \qty{100}{\second \per \mm^{2}} were treated
as a $b=0$ image. MP-PCA denoising as implemented in MRtrix3's
\texttt{dwidenoise}\citep{dwidenoise1} was applied with a 5-voxel window. After
MP-PCA, B1 field inhomogeneity was corrected using \texttt{dwibiascorrect} from
MRtrix3 with the N4 algorithm \citep{n4}. After B1 bias correction, the mean
intensity of the DWI series was adjusted so all the mean intensity of the $b=0$
images matched across each separate DWI scanning sequence.

FSL (version 6.0.3:b862cdd5)'s eddy was used for head motion correction
and Eddy current correction \citep{anderssoneddy}. Eddy was configured
with a \(q\)-space smoothing factor of 10, a total of 5 iterations, and
\num{1000} voxels used to estimate hyperparameters. A linear first level model
and a linear second level model were used to characterize Eddy
current-related spatial distortion. \(q\)-space coordinates were
forcefully assigned to shells. Field offset was attempted to be
separated from participant movement. Shells were aligned post-eddy. Eddy's
outlier replacement was run \citep{eddyrepol}. Data were grouped by
slice, only including values from slices determined to contain at least
\num{250} intracerebral voxels. Groups deviating by more than four standard
deviations from the prediction had their data replaced with imputed
values. Data was collected with reversed phase-encode blips, resulting
in pairs of images with distortions going in opposite directions. Here,
$b=0$ reference images with reversed phase encoding directions were used
along with an equal number of $b=0$ images extracted from the DWI scans.
From these pairs the susceptibility-induced off-resonance field was
estimated using a method similar to that described in \citep{topup}. The
fieldmaps were ultimately incorporated into the Eddy current and head
motion correction interpolation. Final interpolation was performed using
the \texttt{jac} method.

Several confounding time-series were calculated based on the
\emph{preprocessed DWI}: framewise displacement (FD) using the implementation
in \emph{Nipype} following the definitions by \citep{power-fd-dvars}. The DWI
time-series were resampled to ACPC, and their corresponding gradient directions
were rotated accordingly to generate a \emph{preprocessed DWI run
in ACPC space}. 

\end{itemize}

Many internal operations of \emph{QSIPrep} use \emph{Nilearn} 0.6.2
\citep{nilearn}, RRID:SCR\_001362 and \emph{DIPY} \citep{dipy}. For more details
of the pipeline, see
\href{https://qsiprep.readthedocs.io/en/latest/workflows.html}{the section
corresponding to workflows in \emph{QSIPrep}'s documentation}.

\subsection{Cloud-based distributed preprocessing}

The containerization of \emph{QSIPrep} provided a consistent preprocessing
pipeline for each participant but the number of participants made serial processing of
each participant prohibitive on a single machine. We used \emph{cloudknot}, a
previously developed cloud-computing library \citep{cloudknot} to
parallelize the preprocessing over individual participants on spot instances in the
Amazon Web Services Batch service. \emph{Cloudknot} takes as input a
user-defined Python function and creates the necessary AWS infrastructure to map
that function onto a range of inputs, in this case, the participant IDs. The Python
preprocessing function was a thin wrapper around \emph{QSIPrep}'s command line
interface and is provided in a jupyter notebook in the HBN-POD2 GitHub
repository in the ``notebooks'' directory. Using \emph{cloudknot} and AWS Batch
Spot Instances, the preprocessing cost less than \textdollar1.00 per participant.

\subsection{Expert quality control}

The expert QC ``gold standard'' subset was created by randomly selecting 200
participants from the preprocessed dataset, sampled such that the proportional site
distribution in the gold standard subset matched that of the preprocessed
dataset.

We created a web application for expert quality control of preprocessed dMRI,
called \emph{dmriprep-viewer} \citep{richie-halford2021-viewer}. The viewer
ingests \emph{QSIPrep} outputs and generates a browser-based interface for
expert QC. It provides a study overview page displaying the distributions of
\emph{QSIPrep}'s automated QC metrics (described at
\url{https://qsiprep.readthedocs.io/en/latest/preprocessing.html#quality-control-data}).
Each datum on the study overview page is interactively linked to a participant-level
QC page that provides an interactive version of \emph{QSIPrep}'s visual reports
(described at
\url{https://qsiprep.readthedocs.io/en/latest/preprocessing.html#visual-reports}).
The viewer allows users to assign a rating of $-2$ (definitely fail), $-1$
(probably fail), $0$ (not sure), $1$ (probably pass), or $2$ (definitely pass) to a
participant. To standardize rater expectations before rating, expert raters watched
a tutorial video (available on YouTube at \url{https://youtu.be/SQ0v-O-e5b8} and in the OSF project). They then rated each
participant and saved their scores and sent them to the lead author for compilation.

To compute the pairwise Cohen's $\kappa$ scores in Figure~\ref{fig:expert-qc:irr}, we
used the \emph{scikit-learn} \citep{scikit-learn} \texttt{cohen\_kappa\_score}
function with quadratic weights. To compute intra-class correlation, we used the
\emph{pingouin} statistical package \citep{vallat2018pingouin}
\texttt{intraclass\_corr} function. The expert rating analysis can be replicated
using the \texttt{make expert-qc} command in the HBN-POD2 GitHub repository.
The mean expert ratings were scaled to the range \numrange{0}{1}, so that a mean rating from \numrange{0}{0.2} corresponds to an expert rating of ``definitely fail'', a mean rating from \numrange{0.2}{0.4} corresponds to ``probably fail'', from \numrange{0.4}{0.6} corresponds to ``not sure'', from \numrange{0.6}{0.8} corresponds to ``probably pass'', and \numrange{0.8}{1.0} corresponds to ``definitely pass.'' These expert scores are available in the
``expert\_qc\_score'' column of the \texttt{participants.tsv} file on FCP-INDI.

\subsection{Community scientist quality control}

The community science web application is based on the SwipesForScience framework
\url{https://swipesforscience.org/}, which generates a web application for
community science given an open repository of images to be labelled and a
configuration file. The source code for the \emph{Fibr} web application is
available at \url{https://github.com/richford/fibr}. After a brief tutorial,
community scientists provided binary pass/fail ratings based on the DEC-FA from
a fit of a DTI model to each participant's preprocessed dMRI data. These images were
generated using a \emph{DIPY} \citep{dipy} \texttt{TensorModel} in a
\emph{cloudknot}-enabled Jupyter notebook that is available in the ``notebooks''
directory of the \emph{Fibr} GitHub repository. \emph{Fibr} saves each community
rating to its Google Firebase backend, the contents of which have been archived
to the HBN-POD2 OSF project.

The \emph{Fibr} ratings were then combined with 31 automated \emph{QSIPrep} QC
metrics to train the gradient boosted trees models XGB, XGB-f, and XGB-q. See Appendix~\ref{app:feature-importance} for a list of these automated QC metrics and a measure of their global feature importance in the XGB and XGB-q models. These
models were implemented as binary classifiers using the XGBoost library
\citep{xgboost}. The targets for these classifiers were the mean expert ratings
in the gold standard dataset, rescaled to the range $[0, 1]$ and binarized with
a threshold of $0.5$. Using repeated
stratified K-fold cross-validation, with three splits and two repeats, we
evaluated the models' performance in predicting the gold standard ratings. In
each fold, the best model hyperparameters were chosen using the scikit-optimize
\citep{scikit-optimize} \texttt{BayesSearchCV} class. Saved model checkpoints for
each cross-validation split are available in the HBN-POD2 OSF project. Since
each split resulted in a different XGB model and we required a single QC score
to train the deep learning model, we combined the models from each
cross-validation split using a voting classifier, computing a weighted averaged
of the predicted probability of passing from each model, weighted by its
out-of-sample ROC-AUC. This was implemented using scikit-learn's
\texttt{VotingClassifier} class. Treating the voting classifier as another
``expert'' rater, we reassessed the pairwise Cohen's $\kappa$ and ICC scores as
in the expert QC subsection. The community ratings analysis can be replicated
using the \texttt{make community-qc} command in the HBN-POD2 GitHub repository.
The XGB model's positive class probabilities are available in the
``xgb\_qc\_score'' column of the \texttt{participants.tsv} file on FCP-INDI, while
the XGB-q model's positive class probabilities are available in the
``xgb\_qsiprep\_qc\_score'' column.

\subsection{Deep learning to predict quality control}

\begin{figure}[tbp]
\begin{fullwidth}
    \begin{subfigure}[t]{0.6\linewidth}
    \centering
    \includegraphics[width=\linewidth]{deep-learning-qc/model.pdf}
    \caption{Slicing and combining the input channels}
    \label{fig:dl-architecture:complete}
    \end{subfigure}
    \begin{subfigure}[t]{0.4\linewidth}
    \centering
    \includegraphics[width=\linewidth]{deep-learning-qc/image_model.pdf}
    \caption{CNN architecture}
    \label{fig:dl-architecture:cnn}
    \end{subfigure}
    \caption{%
        {\bf Deep learning model architecture}:
        \textbf{(a)} The CNN-i+q model accepts multichannel input that combined
        four imaging channels with a fifth channel containing 31 \emph{QSIPrep}
        automated QC metrics. The imaging channels are separated from the QC
        channel using \texttt{Lambda} layers. The imaging channels are passed
        through a CNN \textbf{(b)}, the output of which is concatenated with the
        QC metrics, batch normalized and passed through two fully-connected (FC)
        layers, with rectified linear unit (ReLu) activation functions and with
        512 and 128 units respectively. Each FC layer is followed by a dropout
        layer which drops 40\% of the input units. The final layer contains a
        single unit with a sigmoid activation function and outputs the
        probability of passing QC.
        %
        \textbf{(b)} The CNN portion of the model passes the imaging input
        through four convolutional blocks. Each block consists of a 3D
        convolutional layer with a kernel size of 3 and a ReLu activation, a 3D
        max pooling layer with a pool size of 2, and a batch normalization layer
        with Tensorflow's default parameters. The number of filters in the
        convolutional layers in each block are 64, 64, 128, and 256 respectively.
        The output of the final block is passed through a 3D global average
        pooling layer with Tensorflow's default parameters.
    }
    \label{fig:dl-architecture}
    \figsupp[Deep learning model loss curves]{%
        {\bf Deep learning model loss curves}:
        The binary cross-entropy loss (top), accuracy (middle), and ROC-AUC
        (bottom) for \textbf{(a)} the CNN-i+q model and \textbf{(a)} the CNN-i
        model. Model performance typically plateaued after twenty epochs but was
        allowed continue until meeting the early stopping criterion. The error
        bands represent a bootstrapped 95\% confidence interval.
    }{%
        \hfill
        \includegraphics[width=0.45\linewidth]{deep-learning-qc/dl_learning_curve_with_qc.pdf}
        \hfill
        \includegraphics[width=0.45\linewidth]{deep-learning-qc/dl_learning_curve_without_qc.pdf}
        \hfill
    }\label{figsupp:dl-loss}
\end{fullwidth}
\end{figure}

The binarized voting classifier's predictions were then used as targets to train a deep
learning binary classifier to predict QC scores based on each participant's preprocessed
dMRI data. We trained two different model architectures
\begin{enumerate*}[%
    label=(\roman*),%
    before=\unskip{: },%
    itemjoin={{, }},%
    itemjoin*={{ and }}]
    \item CNN-i, which took only preprocessed dMRI images as input
    \item CNN-i+q, whose input also included \emph{QSIPrep}'s automated QC metrics.
\end{enumerate*}
Both models were implemented in Tensorflow 2 \citep{tensorflow} using the Keras
module \citep{keras}. The image processing part of the model architecture was
identical for both models: a modification of an existing 3D CNN
\citep{zunair2020-bs} previously applied to assess tuberculosis severity
\citep{dicente2019clef}. It accepts a 3D volume as input with four channels
\begin{enumerate*}[%
    label=(\roman*),%
    before=\unskip{: },%
    itemjoin={{, }},%
    itemjoin*={{ and }}]
    \item the $b=0$ reference volume
    \item DEC-FA in the $x$-direction
    \item DEC-FA in the $y$-direction
    \item DEC-FA in the $z$-direction.
\end{enumerate*}
The \emph{QSIPrep}'s automated QC metrics were included as an additional fifth
channel. The CNN-i+q model architecture is summarized in
Figure~\ref{fig:dl-architecture}. Upon input, the CNN-i+q model extracts the
imaging channels and passes them through the CNN architecture. The remaining QC
metric channel is flattened and passed ``around'' the CNN architecture and
concatenated with the output of the convolutional layers. This concatenated
output is then passed through a fully-connected layer to produce a single
output, the probability of passing QC. This architecture has \num{1438783} trainable
parameters.

We used \emph{DIPY} \citep{dipy} and \emph{cloudknot} \citep{cloudknot} to
generate these multichannel volumes for each participant and save them as NIfTI-1
files \citep{nifti}. These NIfTI files were then converted to the Tensorflow
TFRecord format using the \emph{Nobrainer} deep learning framework
\citep{nobrainer}. The Jupyter notebooks used to create these NIfTI and TFRecord
files are available in the ``notebooks'' directory of the \emph{Fibr} GitHub
repository.

We trained each model using the Google Cloud AI Platform Training service;
the HBN-POD2 GitHub repository contains Docker services to launch training (with
\texttt{make dl-train}) and prediction (with \texttt{make dl-predict}) jobs on
Google Cloud, if the user has provided the appropriate credentials in an
environment file and placed the TFRecord files on Google Cloud Storage. To
estimate the variability in model training, we trained ten separate models using
different training and validation splits of the data. The gold standard dataset
was not included in any of these splits and was reserved for reporting final
model performance. Models were optimized for binary crossentropy loss using the
Adam optimizer \citep{kingma2017adam} with an initial learning rate of 0.0001. We
reduced the learning rate by a factor of 0.5 when the validation loss plateaued
for more than two epochs. We also stopped training when the validation loss
failed to improve by more than 0.001 for twenty consecutive epochs. These two
adjustments were made using the \texttt{ReduceLROnPlateau} and
\texttt{EarlyStopping} callbacks in Tensorflow 2 \citep{tensorflow} respectively.
The training and validation loss curves for both the CNN-i and CNN-i+q models
are depicted in \FIGSUPP[dl-architecture]{dl-loss}. While the CNN-i+q model achieved
better validation loss, it did not outperform the CNN-i model on the held out
gold standard dataset. The CNN-i+q model's positive class probabilities are
available in the ``dl\_qc\_score'' column of the \texttt{participants.tsv} file on
FCP-INDI.

To generate the attribution maps, we followed Tensorflow's integrated gradients
tutorial \citep{integrated-gradients-tutorial} with a black baseline image and
128 steps in the Riemann sum approximation of the integral (i.e.
\texttt{m\_steps = 128}). In the HBN-POD2 GitHub repository, we provide a Docker
service to compute integrated gradient attribution maps on Google Cloud, which
can be invoked using the \texttt{make dl-integrated-gradients} command.

\subsection{Site generalization experiments}

\arhtodo[inline]{Add methods for site generalization}

\subsection{QC bundle profiles}

To generate bundle profiles, reconstruction was performed using the
\emph{QSIprep} 0.12.1 preconfigured reconstruction workflow
\texttt{mrtrix\_multishell\_msmt}, modified to generate two million streamlines
rather than the default ten million. Multi-tissue fiber response functions were
estimated using the dhollander algorithm. Fiber orientation distributions (FODs) were estimated via constrained
spherical deconvolution (CSD, \citep{originalcsd, tournier2008csd}) using an
unsupervised multi-tissue method \citep{dhollander2019response,
dhollander2016unsupervised}. Reconstruction was done using MRtrix3
\citep{mrtrix3}. FODs were intensity-normalized using mtnormalize
\citep{mtnormalize}.

These tractograms were then used as input to the Python Automated Fiber
Quantification toolbox (pyAFQ) \citep{kruper2021evaluating}. Twenty-four major
tracts, which are enumerated in Figure~\ref{fig:qc-profiles:md}, were identified using
multiple criteria: streamlines are selected as candidates for inclusion in a
bundle of streamlines that represents a tract if they pass through known
inclusion ROIs and do not pass through exclusion ROIs \citep{Wakana2007-nw}. In
addition, a probabilistic atlas is used to exclude streamlines which are
unlikely to be part of a tract and to adjudicate in cases where a streamline
could belong to more than one tract \citep{Hua2008-di}. Each streamline is
resampled to 100 nodes and the robust mean at each location is calculated by
estimating the 3D covariance of the location of each node and excluding
streamlines that are more than 5 standard deviations from the mean location in
any node. Finally, a bundle profile of tissue properties in each bundle was
created by interpolating the value of MRI maps of these tissue properties to the
location of the nodes of the resampled streamlines designated to each bundle. In
each of 100 nodes, the values are summed across streamlines, weighting the
contribution of each streamline by the inverse of the mahalanobis distance of
the node from the average of that node across streamlines. This means that
streamlines that are more representative of the tract contribute more to the
bundle profile, relative to streamlines that are on the edge of the tract.

These processes create bundle profiles, in which diffusion measures are
quantified and averaged along twenty-four major fiber tracts. We retain only the
mean diffusivity (MD) and the fractional anisotropy (FA) from a diffusion
kurtosis imaging (DKI) model \citep{jensen2005-ta}, and impute missing bundles
using median imputation as implemented by \emph{scikit-learn}'s
\texttt{SimpleImputer} class. 
Because the HBN-POD2 bundle
profiles exhibit strong site effects \citep{richie-halford2021multidimensional},
we used the ComBat harmonization method to robustly adjust for site effects in
the tract profiles. Initially designed to correct for site effects in gene
expression studies \citep{Johnson2007-kl}, ComBat employs a parametric empirical
Bayes approach to adjust for batch effects and has since been applied to
multi-site cortical thickness measurements \citep{fortin2018-hk}, multi-site DTI
studies \citep{fortin2017-be}, and functional MRI data in the Adolescent Brain
Cognitive Development Study (ABCD) \citep{nielson2018detecting}. We rely on the
\emph{neurocombat\_sklearn} library \citep{neurocombat-sklearn}, to apply ComBat in
before plotting bundle profiles in Figure~\ref{fig:qc-profiles:md} using
plotting functions from the AFQ-Insight package
\citep{richie-halford2019insight}. The bundle profile analysis can be replicated
using the \texttt{make bundle-profiles} command in the HBN-POD2 GitHub
repository.

\subsection{Brain age prediction}

We evaluated the effect of varying the QC cutoff on model performance by
observing cross-validated $R^2$ values of gradient boosted trees models
implemented using XGBoost. The input feature space for each model consisted of
\num{4800} features per participant, comprising 100 nodes for each of MD and FA in
the twenty-four major tracts. We imputed missing bundles and harmonized the
different scanning sites as above. The XGBoost models' hyperparameters were
hand-tuned to values that have been performant in the authors' previous
experience. Within the limited age range of the HBN study, MD and FA follow logarithmic maturation trajectories \cite{yeatman2014lifespan}. We therefore log-transformed each participant's age before prediction using
\emph{scikit-learn}'s \texttt{TransformedTargetRegressor} class. For each value
of the QC cutoff between 0 and 0.95, in steps of 0.05, we computed the
cross-validated $R^2$ values using \emph{scikit-learn}'s
\texttt{cross\_val\_score} function with repeated K-fold cross-validation using
five folds and five repeats.

\section{Author contributions statement}

The last two authors named share senior authorship. The first two authors named
share lead authorship. The remaining authors are listed in alphabetical order,
with the exception of the \emph{Fibr} Community Science Consortium, whose members
provided community science QC ratings.

\alltodo[inline]{Please add your initials here as appropriate}

We describe contributions to the paper using the CRediT taxonomy \citep{brand2015-vd,allen2014-oc}:
\begin{itemize}
    \item Conceptualization: A.R-H., A.R., T.S., and M.C.;
    \item Methodology: A.R-H. and A.R.;
    \item Software: A.R-H. and M.C.;
    \item Validation: ;
    \item Formal Analysis: A.R-H. and M.C.;
    \item Investigation: A.R-H., V.J.S., and M.C.;
    \item Resources: M.M.;
    \item Data curation: S.C., M.C., V.J.S., I.I.K., B.A-P. and L.A.;
    \item Writing – Original Draft: A.R-H. and A.R.;
    \item Writing – Review \& Editing: T.S., V.J.S., I.I.K and B.A-P., S.C.;
    \item Visualization: A.R-H.;
    \item Supervision: A.R. and T.S.;
    \item Project Administration: A.R-H. and A.R.;
    \item Funding Acquisition: A.R. and T.S.
\end{itemize}

\section{Acknowledgments}

We would like to thank Anisha Keshavan for useful discussions of community
science and web-based quality control and for her work on SwipesForScience. This
manuscript was prepared using a limited access dataset obtained from the Child
Mind Institute Biobank, The Healthy Brain Network dataset. This manuscript
reflects the views of the authors and does not necessarily reflect the opinions
or views of the Child Mind Institute. This work was supported via BRAIN Initiative grant 1RF1MH121868-01 from the National Institutes of Mental Health. Additional support was provided by R01MH120482 and the Penn/CHOP Lifespan Brain Institute. 

\bibliography{hbn-pod2}

%%%%%%%%%%%%%%%%%%%%%%%%%%%%%%%%%%%%%%%%%%%%%%%%%%%%%%%%%%%%
%%% APPENDICES
%%%%%%%%%%%%%%%%%%%%%%%%%%%%%%%%%%%%%%%%%%%%%%%%%%%%%%%%%%%%

\appendix
\begin{appendixbox}
\label{app:feature-importance}
\section{XGB feature importance}

SHAP is a method to explain individual predictions based on game theoretically optimal Shapley values \cite{lundberg2017unified}. To estimate global feature importance for the XGB and XGB-q models, we use the \texttt{shap} library's \texttt{TreeExplainer} \cite{lundberg2020local} and average the absolute Shapley value per feature across each individual prediction. Tables~\ref{tab:xgb-shap} and \ref{tab:xgb-q-shap} list the \emph{QSIPrep} automated QC metric features in order of decreasing mean absolute shap value for the XGB and XGB-q models, respectively. We chose the top three metrics from Table~\ref{tab:xgb-shap} to plot metric distributions in Figure~\ref{fig:metric-dist} and correlations with the expert QC results in Figure~\ref{fig:expert-qc}.

\begin{multicols}{2}
\begin{tabular}{lr|}
\toprule
{} &  mean abs shap \\
feature               &                           \\
\midrule
raw\_neighbor\_corr     &                  0.666429 \\
max\_rel\_translation   &                  0.348662 \\
raw\_num\_bad\_slices    &                  0.288937 \\
t1\_neighbor\_corr      &                  0.282198 \\
raw\_incoherence\_index &                  0.229733 \\
raw\_coherence\_index   &                  0.162103 \\
max\_rel\_rotation      &                  0.118963 \\
mean\_fd               &                  0.116457 \\
max\_fd                &                  0.099359 \\
max\_rotation          &                  0.078774 \\
t1\_coherence\_index    &                  0.035553 \\
t1\_dice\_distance      &                  0.034510 \\
max\_translation       &                  0.032323 \\
t1\_incoherence\_index  &                  0.030225 \\
raw\_voxel\_size\_x      &                  0.000000 \\
raw\_voxel\_size\_y      &                  0.000000 \\
raw\_voxel\_size\_z      &                  0.000000 \\
raw\_num\_directions    &                  0.000000 \\
raw\_max\_b             &                  0.000000 \\
raw\_dimension\_y       &                  0.000000 \\
raw\_dimension\_z       &                  0.000000 \\
t1\_voxel\_size\_x       &                  0.000000 \\
t1\_dimension\_x        &                  0.000000 \\
t1\_dimension\_y        &                  0.000000 \\
t1\_dimension\_z        &                  0.000000 \\
t1\_voxel\_size\_y       &                  0.000000 \\
t1\_voxel\_size\_z       &                  0.000000 \\
t1\_max\_b              &                  0.000000 \\
t1\_num\_bad\_slices     &                  0.000000 \\
t1\_num\_directions     &                  0.000000 \\
raw\_dimension\_x       &                  0.000000 \\
\bottomrule
\end{tabular}
\captionof{table}{%
  XGB mean absolute shap values
  \label{tab:xgb-shap}
}

{\nolinenumbers
\begin{tabular}{|lr}
\toprule
{} &  mean abs shap \\
feature               &                           \\
\midrule
raw\_neighbor\_corr     &                  0.767536 \\
raw\_incoherence\_index &                  0.453897 \\
raw\_num\_bad\_slices    &                  0.430422 \\
t1\_coherence\_index    &                  0.382218 \\
max\_rel\_translation   &                  0.363052 \\
raw\_coherence\_index   &                  0.320438 \\
t1\_neighbor\_corr      &                  0.250948 \\
t1\_dice\_distance      &                  0.248104 \\
t1\_incoherence\_index  &                  0.242348 \\
max\_rel\_rotation      &                  0.135590 \\
mean\_fd               &                  0.128642 \\
max\_translation       &                  0.120815 \\
max\_fd                &                  0.119739 \\
max\_rotation          &                  0.101209 \\
t1\_num\_bad\_slices     &                  0.007075 \\
raw\_dimension\_y       &                  0.000000 \\
raw\_dimension\_z       &                  0.000000 \\
raw\_voxel\_size\_x      &                  0.000000 \\
raw\_voxel\_size\_y      &                  0.000000 \\
raw\_voxel\_size\_z      &                  0.000000 \\
raw\_max\_b             &                  0.000000 \\
t1\_voxel\_size\_x       &                  0.000000 \\
raw\_num\_directions    &                  0.000000 \\
t1\_dimension\_x        &                  0.000000 \\
t1\_dimension\_y        &                  0.000000 \\
t1\_dimension\_z        &                  0.000000 \\
t1\_voxel\_size\_y       &                  0.000000 \\
t1\_voxel\_size\_z       &                  0.000000 \\
t1\_max\_b              &                  0.000000 \\
t1\_num\_directions     &                  0.000000 \\
raw\_dimension\_x       &                  0.000000 \\
\bottomrule
\end{tabular}
\captionof{table}{%
  XGB-q mean absolute shap values
  \label{tab:xgb-q-shap}
}
}
\end{multicols}
\end{appendixbox}

\appendix
\begin{appendixbox}
\label{app:variants}
\section{CuBIDS variant annotation}

We identified 20 unique DWI acquisitions across HBN-POD2, which are summarized in Table~\ref{tab:variants}. Site CBIC has two acquisition types: ``64dir,'' which shares it's pulse sequence with sites RU and CUNY, and ``ABCD64dir,'' with acquisition parameters that better match the ABCD study (TE=\qty{0.089}{\second} and TR=\qty{4.1}{\second}). The ``Most\_Common'' variant identifies the most common combination of acquisition parameters for a given site and acquisition. The ``Low\_Volume'' variant identifies participants from all sites with less that 129 DWI volumes, which is the number of volumes in the most common variants. All remaining variants names identify the acquisition parameter(s) that differ from those of the most common variant. For example, the ``MultibandAccelerationFactor'' variant has a different multiband acceleration factor than that of the the most common variant but all participants within that variant share the same multiband acceleration factor. Variants that differ by multiple acquisition parameters have names that are composed of concatenated parameters. For example, the variant ``Dim3SizeVoxelSizeDim3'' varies both in the number of voxels in dimension 3 (``Dim3Size'') and in the voxel size in dimension 3 (``VoxelSizeDim3'').


\begin{tabular}{lllr}
\toprule
     Site &      Acquisition &                        Variant &  Count \\
\midrule
     CBIC &            64dir &                    Most\_Common &    828 \\
     CBIC &            64dir &                      Obliquity &     32 \\
     CBIC &            64dir &     VoxelSizeDim1VoxelSizeDim2 &      1 \\
     CBIC &        ABCD64dir &                    Most\_Common &     15 \\
     CBIC &        ABCD64dir &                        HasFmap &      2 \\
     CBIC &        ABCD64dir &    MultibandAccelerationFactor &      1 \\
     CBIC &        ABCD64dir &                      Obliquity &      1 \\
     CUNY &            64dir &                    Most\_Common &     68 \\
     CUNY &            64dir &          Dim3SizeVoxelSizeDim3 &      4 \\
     CUNY &            64dir &                      Obliquity &      2 \\
       RU &            64dir &                    Most\_Common &    859 \\
       RU &            64dir &                         NoFmap &      5 \\
       RU &            64dir &                      Obliquity &      8 \\
       RU &            64dir &         PhaseEncodingDirection &      1 \\
       SI &            64dir &                       EchoTime &      1 \\
       SI &            64dir & EchoTimePhaseEncodingDirection &      9 \\
       SI &            64dir &                    Most\_Common &    269 \\
       SI &            64dir &                         NoFmap &      2 \\
       SI &            64dir &                      Obliquity &     12 \\
All Sites & All Acquisitions &               Low\_Volume\_Count &     14 \\
\bottomrule
\end{tabular}
\captionof{table}{%
  Participant counts for HBN-POD2 variants.
  \label{tab:variants}
}

\end{appendixbox}

\end{document}
