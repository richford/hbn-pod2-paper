\documentclass[fleqn,10pt]{wlscirep}
\usepackage[utf8]{inputenc}
\usepackage[T1]{fontenc}
\usepackage{todonotes}

\title{A preprocessed open diffusion derivatives dataset from the Healthy Brain Network}

\author[1,*$\dagger$]{Adam Richie-Halford}
\author[2,$\dagger$]{Matthew Cieslak}
\author[4]{Lei Ai}
\author[5]{Sendy Caffarra}
\author[4]{Alexandre R. Franco}
\author[5]{Iliana Karipidis}
\author[3]{John Kruper}
\author[5]{Barbara Avelar Pereira}
\author[5]{Ethan Roy}
\author[2]{Valerie J. Sydnor}
\author[5]{Jason Yeatman}
\author[4]{Michael Milham}
\author[6]{Fibr user 1}
\author[6]{Fibr user 2}
\author[6]{Fibr user 3}
\author[6]{Fibr user 4}
\author[6]{Fibr user 5}
\author[6]{Fibr user 6}
\author[6]{Fibr user 7}
\author[6]{Fibr user 8}
\author[6]{Fibr user 9}
\author[6]{Fibr user 10}
\author[6]{Fibr user 11}
\author[6]{Fibr user 12}
\author[6]{Fibr user 13}
\author[6]{Fibr user 14}
\author[6]{Fibr user 15}
\author[6]{Fibr user 16}
\author[6]{Fibr user 17}
\author[6]{Fibr user 18}
\author[6]{Fibr user 19}
\author[6]{Fibr user 20}
\author[6]{Fibr user 21}
\author[6]{Fibr user 22}
\author[6]{Fibr user 23}
\author[6]{Fibr user 24}
\author[6]{Fibr user 25}
\author[6]{Fibr user 26}
\author[6]{Fibr user 27}
\author[6]{Fibr user 28}
\author[6]{Fibr user 29}
\author[6]{Fibr user 30}
\author[6]{Fibr user 31}
\author[6]{Fibr user 32}
\author[6]{Fibr user 33}
\author[6]{Fibr user 34}
\author[6]{Fibr user 35}
\author[6]{Fibr user 36}
\author[6]{Fibr user 37}
\author[6]{Fibr user 38}
\author[6]{Fibr user 39}
\author[6]{Fibr user 40}
\author[6]{Fibr user 41}
\author[6]{Fibr user 42}
\author[6]{Fibr user 43}
\author[6]{Fibr user 44}
\author[6]{Fibr user 45}
\author[6]{Fibr user 46}
\author[6]{Fibr user 47}
\author[6]{Fibr user 48}
\author[6]{Fibr user 49}
\author[6]{Fibr user 50}
\author[6]{Fibr user 51}
\author[6]{Fibr user 52}
\author[6]{Fibr user 53}
\author[6]{Fibr user 54}
\author[6]{Fibr user 55}
\author[6]{Fibr user 56}
\author[6]{Fibr user 57}
\author[6]{Fibr user 58}
\author[6]{Fibr user 59}
\author[6]{Fibr user 60}
\author[6]{Fibr user 61}
\author[6]{Fibr user 62}
\author[6]{Fibr user 63}
\author[6]{Fibr user 64}
\author[6]{Fibr user 65}
\author[6]{Fibr user 66}
\author[6]{Fibr user 67}
\author[6]{Fibr user 68}
\author[6]{Fibr user 69}
\author[6]{Fibr user 70}
\author[6]{Fibr user 71}
\author[6]{Fibr user 72}
\author[6]{Fibr user 73}
\author[6]{Fibr user 74}
\author[6]{Fibr user 75}
\author[6]{Fibr user 76}
\author[6]{Fibr user 77}
\author[6]{Fibr user 78}
\author[6]{Fibr user 79}
\author[6]{Fibr user 80}
\author[6]{Fibr user 81}
\author[6]{Fibr user 82}
\author[6]{Fibr user 83}
\author[6]{Fibr user 84}
\author[6]{Fibr user 85}
\author[6]{Fibr user 86}
\author[6]{Fibr user 87}
\author[6]{Fibr user 88}
\author[6]{Fibr user 89}
\author[6]{Fibr user 90}
\author[6]{Fibr user 91}
\author[6]{Fibr user 92}
\author[6]{Fibr user 93}
\author[6]{Fibr user 94}
\author[6]{Fibr user 95}
\author[6]{Fibr user 96}
\author[6]{Fibr user 97}
\author[6]{Fibr user 98}
\author[6]{Fibr user 99}
\author[6]{Fibr user 100}
\author[6]{Fibr user 101}
\author[6]{Fibr user 102}
\author[6]{Fibr user 103}
\author[6]{Fibr user 104}
\author[6]{Fibr user 105}
\author[6]{Fibr user 106}
\author[6]{Fibr user 107}
\author[6]{Fibr user 108}
\author[6]{Fibr user 109}
\author[6]{Fibr user 110}
\author[6]{Fibr user 111}
\author[6]{Fibr user 112}
\author[6]{Fibr user 113}
\author[6]{Fibr user 114}
\author[6]{Fibr user 115}
\author[6]{Fibr user 116}
\author[6]{Fibr user 117}
\author[6]{Fibr user 118}
\author[6]{Fibr user 119}
\author[6]{Fibr user 120}
\author[6]{Fibr user 121}
\author[6]{Fibr user 122}
\author[6]{Fibr user 123}
\author[6]{Fibr user 124}
\author[6]{Fibr user 125}
\author[6]{Fibr user 126}
\author[6]{Fibr user 127}
\author[6]{Fibr user 128}
\author[6]{Fibr user 129}
\author[6]{Fibr user 130}
\author[6]{Fibr user 131}
\author[6]{Fibr user 132}
\author[6]{Fibr user 133}
\author[2,+]{Theodore D. Satterthwaite}
\author[3,1,+]{Ariel Rokem}

\affil[1]{University of Washington, eScience Institute, Seattle, Washington, 98195, USA}
\affil[2]{University of Pennsylvania, Department of Psychiatry, Philadelphia, Pennsylvania, 19104, USA}
\affil[3]{University of Washington, Department of Psychology, Seattle, Washington, 98195, USA}
\affil[4]{Child Mind Institute, New York City, 10022, USA}
\affil[5]{Stanford University, Graduate School of Education and Division of Developmental and Behavioral Pediatrics, Stanford, California, 94305, USA}
\affil[6]{The Fibr Community Science Consortium}

\affil[*]{richford@uw.edu}
\affil[$\dagger$]{these authors contributed equally to this work}
\affil[+]{these authors contributed equally to this work}

%\keywords{Keyword1, Keyword2, Keyword3}

\begin{abstract}
\todo[inline]{Write abstract}
\end{abstract}
\begin{document}

\flushbottom
\maketitle
\thispagestyle{empty}

\section*{Introduction}

The Healthy Brain Network (HBN) is a landmark pediatric mental health study collecting MRI images and clinical assessment data from 10,000 New York City area children and adolescents (Alexander 2017). The HBN dataset contains a wealth of phenotypic and imaging data, including diffusion MRI (dMRI) data, which allows for analysis of the physical properties of developing white matter (Wandell 2016). This dMRI data is openly available in raw form through the Functional Connectomes Project and the International Neuroimaging Data-Sharing Initiative (FCP-INDI), spurring collaboration on open big-data reproducible science (Avesani 2019). However, analysis of dMRI data must start with a pipeline of critical preprocessing steps, such as eddy current correction, motion correction, and adjustment of the gradient directions. Because of the complexity of some of these steps, investigators may neglect to perform preprocessing or may make errors that can induce bias in their subsequent interpretation of the data (Jones 2010). Furthermore, once preprocessing is done correctly and transparently once, there is little need for researchers to repeat this step. Thus, there is a need for an openly available preprocessed diffusion derivative dataset that applies best practices in preprocessing in a robust and transparent way (Cieslak 2020). Accordingly, here we introduce the HBN Preprocessed Open Diffusion Derivatives (HBN-POD2), a large dataset for the analysis of structural brain connectivity and pediatric mental health.

\section*{Results}

Outputs from each of the curation and preprocessing steps, as well as QC ratings, are openly available on FCP-INDI. Similarly, HBN-POD2 offers open QC metrics, eliminating the need for investigators to repeat quality assessment for over 2,000 scans. Using a previously developed cloud-computing library 7 to parallelize the preprocessing over individual subjects on spot instances in the Amazon Web Services Batch service, this dMRI preprocessing cost less than \textdollar1.00 per subject.

\section*{Discussion}

We present HBN-POD2, one of the largest youth diffusion imaging datasets with derived measures currently available. It is openly available and complies with the current draft of the BIDS diffusion derivative specification. It will grow continuously as the HBN study acquires more data, eventually reaching its 10,000 subject goal. The data is amenable to many different analyses, including tractometry (Yeatman 2012), graph theoretical analysis (Yeh 2020), and combinations with functional data for the same subjects. The availability of standardized preprocessed diffusion data will allow researchers to create and test hypotheses on the white matter properties underlying behavior and disease, from reading and math acquisition to childhood adversity and mental health. As such, this dataset will accelerate discovery at the nexus of structural connectivity and neurodevelopmental and learning disorders.

\section*{Methods}

We curated the imaging metadata for 2,615 of the 2,747 currently available HBN subjects. Using dcm2bids and custom scripts, we conformed the data to the Brain Imaging Data Structure (BIDS; Gorgolewski 2016) specification. We then performed dMRI preprocessing on 2,136 subjects, using QSIPrep (Cieslak 2020), a robust and scalable pipeline to group, distortion correct, motion correct, denoise, coregister and resample MRI scans (see Figure 1). In total, 417 subjects failed this preprocessing step, largely due to missing dMRI files. To assess image quality, we released a citizen science web application, drawing on the success of a previous application in assessing the quality of HBN's structural MRI data (Keshavan 2019). After a brief tutorial, citizen scientists provided binary pass/fail ratings based on the directionally-colorized fractional anisotropy from DTI of each subject's preprocessed dMRI data. These citizen scientist ratings were then combined with expert ratings (see Figure 2) to train a neural network to assign a quality control (QC) rating to each subject.

\subsection*{BIDS curation}
\subsection*{Diffusion preprocessing}
\subsection*{Expert quality control}
\subsection*{Community scientist quality control}
\subsection*{Deep learning to predict quality control}
\subsection*{Brain age prediction}

\bibliography{hbn-pod2}

\section*{Acknowledgements}

Acknowledgements should be brief, and should not include thanks to anonymous referees and editors, or effusive comments. Grant or contribution numbers may be acknowledged.

\section*{Author contributions statement}

Must include all authors, identified by initials, for example:
A.A. conceived the experiment(s),  A.A. and B.A. conducted the experiment(s), C.A. and D.A. analysed the results.  All authors reviewed the manuscript. 

\section*{Additional information}

To include, in this order: \textbf{Accession codes} (where applicable); \textbf{Competing interests} (mandatory statement). 

The corresponding author is responsible for submitting a \href{http://www.nature.com/srep/policies/index.html#competing}{competing interests statement} on behalf of all authors of the paper. This statement must be included in the submitted article file.

% \begin{figure}[ht]
% \centering
% \includegraphics[width=\linewidth]{stream}
% \caption{Legend (350 words max). Example legend text.}
% \label{fig:stream}
% \end{figure}

% \begin{table}[ht]
% \centering
% \begin{tabular}{|l|l|l|}
% \hline
% Condition & n & p \\
% \hline
% A & 5 & 0.1 \\
% \hline
% B & 10 & 0.01 \\
% \hline
% \end{tabular}
% \caption{\label{tab:example}Legend (350 words max). Example legend text.}
% \end{table}

% Figures and tables can be referenced in LaTeX using the ref command, e.g. Figure \ref{fig:stream} and Table \ref{tab:example}.

\end{document}